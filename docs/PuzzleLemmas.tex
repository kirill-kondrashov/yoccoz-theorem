\section{Puzzle Lemmas}
\label{sec:PuzzleLemmas}

This section describes the verification of the file \texttt{Mlc/Quadratic/Complex/PuzzleLemmas.lean}, which establishes basic properties of the dynamical puzzle pieces defined in \texttt{Puzzle.tex}.

\subsection{Nesting of Puzzle Pieces}

A fundamental property of the puzzle construction is that pieces at deeper levels are contained in pieces at shallower levels.

\begin{lstlisting}[language=Lean]
theorem dynamical_puzzle_piece_nested (c : ℂ) (n : ℕ) :
    DynamicalPuzzlePiece c (n + 1) 0 ⊆ DynamicalPuzzlePiece c n 0 := by
  apply connectedComponentIn_mono
  intro w hw
  dsimp at *
  -- Proof idea: The level sets are defined by `G(w) < 1/2^n`.
  -- Since `1/2^{n+1} < 1/2^n`, the set for `n+1` is contained in the set for `n`.
  -- Connected component of a subset is contained in the connected component of the superset.
  apply lt_trans hw
  rw [pow_succ]
  nth_rw 2 [← one_mul ((1 / 2 : ℝ) ^ n)]
  rw [mul_comm]
  apply mul_lt_mul_of_pos_right
  · norm_num
  · apply pow_pos
    norm_num
\end{lstlisting}

\noindent
\textbf{Detailed Explanation:}
\begin{itemize}
    \item \textbf{Goal}: Show $P_{n+1}(0) \subseteq P_n(0)$.
    \item \texttt{connectedComponentIn\_mono}: This is a topological lemma stating that if $A \subseteq B$, then the component of $A$ containing $x$ is a subset of the component of $B$ containing $x$.
    \item \textbf{Set containment}: The core check reduces to showing $\{w \mid G(w) < (1/2)^{n+1}\} \subseteq \{w \mid G(w) < (1/2)^n\}$.
    \item \texttt{lt\_trans}: We have $G(w) < (1/2)^{n+1}$ and we need to show $G(w) < (1/2)^n$. This follows if $(1/2)^{n+1} < (1/2)^n$.
    \item \texttt{mul\_lt\_mul\_of\_pos\_right}: The inequality $(1/2) \cdot (1/2)^n < 1 \cdot (1/2)^n$ holds since $1/2 < 1$.
\end{itemize}

\begin{proof}
Since $(1/2)^{n+1}<(1/2)^n$ the defining sublevel for depth $n+1$ is contained in that for depth $n$, and \texttt{connectedComponentIn\_mono} yields the inclusion of components; hence $P_{n+1}(0)\subseteq P_n(0)$.
\end{proof}

\subsection{Membership of the Critical Point}

For parameters in the Mandelbrot set, the critical point 0 is always contained in its puzzle piece (which is defined \textbf{around} 0, but this confirms the set is non-empty and well-defined).

\begin{lstlisting}[language=Lean]
theorem mem_dynamical_puzzle_piece_self (c : ℂ) (hc : c ∈ MandelbrotSet) (n : ℕ) :
    0 ∈ DynamicalPuzzlePiece c n 0 := by
  -- Proof idea: Since `c ∈ M`, `0 ∈ K(c)`.
  -- Points in `K(c)` have Green's function 0.
  -- Since `0 < 1/2^n`, `0` satisfies the condition `G(0) < 1/2^n`.
  -- Thus `0` is in the connected component of the level set containing `0`.
  have h0 : 0 ∈ K c := hc
  rw [← green_function_eq_zero_iff_mem_K] at h0
  dsimp [DynamicalPuzzlePiece]
  apply mem_connectedComponentIn
  change green_function c 0 < (1 / 2) ^ n
  rw [h0]
  apply pow_pos
  norm_num
\end{lstlisting}

\noindent
\textbf{Detailed Explanation:}
\begin{itemize}
    \item \textbf{Hypothesis}: $c \in \mathcal{M}$.
    \item \texttt{hc}: Implies $0 \in K_c$.
    \item \texttt{green\_function\_eq\_zero\_iff\_mem\_K}: Therefore $G_c(0) = 0$.
    \item \texttt{mem\_connectedComponentIn}: To show $z \in \text{Component}(S, z)$, it suffices to show $z \in S$.
    \item \textbf{Inequality}: We need $G_c(0) < (1/2)^n$. Since $G_c(0)=0$ and $(1/2)^n > 0$, this is always true.
\end{itemize}

\begin{proof}
Because $c\in\mathcal{M}$ we have $0\in K_c$ and hence $G_c(0)=0$. Since $0<(1/2)^n$ it follows that $0$ belongs to the defining sublevel set and therefore to its connected component, proving $0\in P_n(0)$.
\end{proof}

\subsection{Behavior Outside the Mandelbrot Set}

For parameters outside the Mandelbrot set, the critical point eventually escapes any puzzle piece of sufficiently high depth (because the potential $G_c(0)$ is strictly positive).

\begin{lstlisting}[language=Lean]
theorem dynamical_puzzle_piece_empty_of_large_n (c : ℂ) (hc : c ∉ MandelbrotSet) :
    ∃ N, ∀ n ≥ N, 0 ∉ DynamicalPuzzlePiece c n 0 := by
  -- Proof idea: If `c ∉ M`, then `0 ∉ K(c)`, so `G(0) > 0`.
  -- Since `1/2^n → 0`, eventually `1/2^n < G(0)`.
  -- At that point, `0` is no longer in the set `{w | G(w) < 1/2^n}`.
  have h_not_in_K : 0 ∉ K c := hc
  rw [← green_function_pos_iff_not_mem_K] at h_not_in_K
  have h_pos : 0 < green_function c 0 := h_not_in_K

  obtain ⟨N, hN⟩ : ∃ N : ℕ, (1 / 2 : ℝ) ^ N < green_function c 0 := by
    have h_tendsto : Tendsto (fun n : ℕ => (1 / 2 : ℝ) ^ n) atTop ($\mathcal{N}$ 0) := by
      apply tendsto_pow_atTop_nhds_zero_of_lt_one
      · norm_num
      · norm_num
    exact ((tendsto_order.1 h_tendsto).2 (green_function c 0) h_pos).exists

  use N
  intro n hn
  dsimp [DynamicalPuzzlePiece]
  intro h_mem
  have h_in_set : 0 ∈ {w | green_function c w < (1 / 2) ^ n} :=
    (connectedComponentIn_subset {w | green_function c w < (1 / 2) ^ n} 0) h_mem
  rw [mem_setOf_eq] at h_in_set
  have h_le : (1 / 2 : ℝ) ^ n ≤ (1 / 2 : ℝ) ^ N := by
    apply pow_le_pow_of_le_one
    · norm_num
    · norm_num
    · exact hn
  have h_lt : (1 / 2 : ℝ) ^ n < green_function c 0 :=
    lt_of_le_of_lt h_le hN
  linarith
\end{lstlisting}

\noindent
\textbf{Detailed Explanation:}
\begin{itemize}
    \item \textbf{Hypothesis}: $c \notin \mathcal{M}$, which implies $G_c(0) > 0$ (\texttt{h\_pos}).
    \item \texttt{tendsto}: Since $(1/2)^n \to 0$ as $n \to \infty$, there exists some $N$ such that $(1/2)^N < G_c(0)$.
    \item \texttt{use N}: This $N$ works for all $n \ge N$.
    \item \textbf{Contradiction check}: If $0$ were in the puzzle piece $P_n(0)$, then by definition $G_c(0) < (1/2)^n$.
    \item But for $n \ge N$, $(1/2)^n \le (1/2)^N < G_c(0)$.
    \item This gives $G_c(0) < G_c(0)$, a contradiction (\texttt{linarith}).
\end{itemize}

\begin{proof}
If $c\notin\mathcal{M}$ then $G_c(0)>0$ and $(1/2)^n\to0$, so choose $N$ with $(1/2)^N<G_c(0)$. For any $n\ge N$ the condition $G_c(0)<(1/2)^n$ fails, hence $0\notin P_n(0)$ for all such $n$, as required.
\end{proof}
