\section{Puzzle Definitions}
\label{sec:Puzzle}

This section describes the verification of the file \texttt{Mlc/Quadratic/Complex/Puzzle.lean}, which defines the fundamental geometric objects used in the proof: dynamical and parameter puzzle pieces.

\subsection{Dynamical Puzzle Pieces}

The Yoccoz puzzle relies on decomposing the dynamical plane using the level sets of the Green's function.

\begin{lstlisting}[language=Lean]
def DynamicalPuzzlePiece (c : ℂ) (n : ℕ) (z : ℂ) : Set ℂ :=
  connectedComponentIn {w | green_function c w < (1 / 2) ^ n} z
\end{lstlisting}

\noindent
\textbf{Detailed Explanation:}
\begin{itemize}
    \item \texttt{DynamicalPuzzlePiece}: This function defines a set in the complex plane for a given parameter $c$, depth $n$, and target point $z$.
    \item \texttt{green\_function c w < (1 / 2) \^{} n}: This defines the sublevel set of the potential. The threshold $1/2^n$ is standard for Yoccoz puzzles.
    \item \texttt{connectedComponentIn ... z}: We take only the connected component containing $z$. This ensures that puzzle pieces are topological disks (or at least connected), which is crucial for defining their moduli.
\end{itemize}

\subsection{Puzzle Annuli}

We define the annuli formed by the difference of consecutive puzzle pieces around the critical point.

\begin{lstlisting}[language=Lean]
def PuzzleAnnulus (c : ℂ) (n : ℕ) : Set ℂ :=
  DynamicalPuzzlePiece c n 0 \ DynamicalPuzzlePiece c (n + 1) 0
\end{lstlisting}

\noindent
\textbf{Detailed Explanation:}
\begin{itemize}
    \item \texttt{PuzzleAnnulus}: The set difference between the puzzle piece at depth $n$ and the piece at depth $n+1$, both centered at the critical point $0$.
    \item These sets form the "thick" regions whose moduli we will sum up in the application of Grötzsch's inequality.
\end{itemize}

\subsection{Parameter Puzzle Pieces}

The definitions in the dynamical plane are transferred to the parameter plane.

\begin{lstlisting}[language=Lean]
def ParaPuzzlePiece (n : ℕ) : Set ℂ := {c | c ∈ DynamicalPuzzlePiece c n 0}
\end{lstlisting}

\noindent
\textbf{Detailed Explanation:}
\begin{itemize}
    \item \texttt{ParaPuzzlePiece}: The set of parameters $c$ such that $c$ itself is contained in the dynamical puzzle piece of depth $n$ (specifically the one containing 0).
    \item Note: Since $f_c(0) = c$, checking if $c \in P_n(0)$ is checking if the critical value is in the puzzle piece of the critical point.
\end{itemize}
