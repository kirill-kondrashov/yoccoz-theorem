\section{Grötzsch's Inequality and Criterion}
\label{sec:Groetzsch}

This section details the analytical core of the proof: Grötzsch's inequality. This classical result from complex analysis connects the geometry of nested annuli to the convergence properties of their moduli, providing a rigorous criterion for a sequence of sets to shrink to a single point.

\subsection*{Conformal Modulus}

The conformal modulus is a fundamental invariant that measures the "thickness" of an annulus. Geometrically, it generalizes the aspect ratio of a rectangle.

\textbf{Definition} (\texttt{modulus}).
For an annulus $A \subset \mathbb{C}$, the modulus $\operatorname{mod}(A)$ is a non-negative real number.
\begin{itemize}
    \item For a standard annulus $A(r, R) = \{z : r < |z| < R\}$, $\operatorname{mod}(A) = \frac{1}{2\pi} \log(R/r)$.
    \item It is a conformal invariant: biholomorphic maps preserve modulus.
    \item For degenerate sets (like the empty set), $\operatorname{mod}(\emptyset) = 0$.
\end{itemize}

\subsection*{Grötzsch's Inequality}

The power of the modulus comes from its behavior under disjoint union. Just as area is additive, modulus is \textit{superadditive} for nested disjoint annuli.

\textbf{Axiom} (\texttt{groetzsch\_inequality}).
\textit{Let $S \subset \mathbb{C}$ be a domain. If $A, B \subset S$ are disjoint annuli essentially embedded in $S$ (separating the inner and outer boundaries), then:
\[ \operatorname{mod}(A) + \operatorname{mod}(B) \le \operatorname{mod}(S). \]
}

\begin{proof}[Geometric Intuition]
This inequality expresses that we cannot pack arbitrary "conformal thickness" into a finite container. If we stack disjoint annuli inside a larger annulus $S$, their total thickness (modulus) cannot exceed the thickness of $S$.
\end{proof}

\subsection*{The Summability Theorem}

We apply this inequality to a sequence of nested connected sets $P_0 \supset P_1 \supset P_2 \supset \dots$ containing $0$. These sets form the "puzzle pieces" of our construction. We are interested in the annuli formed by their differences: $A_n = P_n \setminus P_{n+1}$.

\textbf{Theorem} (\texttt{modulus\_summable\_of\_nontrivial\_intersection}).
\textit{If the intersection $K = \bigcap_{n} P_n$ is non-trivial (contains more than just $\{0\}$), then the sum of the moduli converges:
\[ \sum_{n=0}^{\infty} \operatorname{mod}(A_n) < \infty. \]
}

\begin{proof}[Proof Idea]
(See Lean: \texttt{modulus\_summable\_of\_nontrivial\_intersection})

If $K$ is non-trivial, it serves as a "hard core" inside all $P_n$. The annuli $A_n$ are thus all contained in the fixed domain $P_0 \setminus K$.
Since the $A_n$ are disjoint, Grötzsch's inequality implies that for any $N$:
\[ \sum_{n=0}^N \operatorname{mod}(A_n) \le \operatorname{mod}(P_0 \setminus K). \]
Since $K$ contains points other than $0$, the container $P_0 \setminus K$ is not the punctured plane (which would have infinite modulus) but a domain with finite modulus. Therefore, the partial sums are bounded, and the series converges.
\end{proof}

\begin{figure}[ht]
    \centering
    \includegraphics[width=0.9\textwidth]{images/groetzsch_viz.png}
    \caption{Geometric Interpretation. Left: A non-trivial core $K$ (red) limits the available space, forcing the moduli of the nested annuli $A_n$ to decay rapidly ($\sum < \infty$). Right: If the intersection is a single point $\{0\}$, there is no inner barrier, allowing the moduli sum to diverge ($\sum = \infty$).}
    \label{fig:groetzsch}
\end{figure}

\subsection*{Grötzsch's Criterion}

By taking the contrapositive of the summability theorem, we obtain a sufficient condition for the intersection to be a single point. This is the tool we use to prove Yoccoz's theorem.

\textbf{Theorem} (\texttt{groetzsch\_criterion}).
\textit{Let $\{P_n\}$ be a nested sequence of connected sets containing $0$. If the sum of the moduli of the difference annuli diverges:
\[ \sum_{n=0}^{\infty} \operatorname{mod}(P_n \setminus P_{n+1}) = \infty, \]
then the intersection consists only of the point $0$:
\[ \bigcap_{n=0}^{\infty} P_n = \{0\}. \]
}

\begin{proof}[Proof Idea]
(See Lean: \texttt{groetzsch\_criterion})

Assume for contradiction that the intersection is not $\{0\}$. Since $0$ is in every set, the intersection must contain some other point $z \ne 0$. This implies the intersection is non-trivial.
By the Summability Theorem above, a non-trivial intersection implies that $\sum \operatorname{mod}(A_n) < \infty$.
But this contradicts the hypothesis that the sum is infinite.
Therefore, the intersection must be exactly $\{0\}$.
\end{proof}

\subsection*{Significance for MLC}
This criterion bridges the gap between combinatorics and topology:
\begin{itemize}
    \item \textbf{Combinatorics}: We will show that for non-renormalizable parameters, the puzzle pieces recur frequently enough to generate infinite total modulus.
    \item \textbf{Topology}: Grötzsch's Criterion then guarantees that the pieces shrink to a point, proving local connectivity at the critical parameter $c$.
\end{itemize}

