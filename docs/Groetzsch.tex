\section{Groetzsch}
\textit{Source: \texttt{Quadratic/Complex/Groetzsch.lean}}

This module contains the formal definitions and proofs.
\subsection*{modulus\_empty}\label{modulus_empty}
The conformal modulus of an annulus.
    We treat this as an opaque function for non-empty sets, but explicitly define it as 0 for the empty set.
    See: [Milnor, Dynamics in One Complex Variable, Appendix B] <https://arxiv.org/pdf/math/9201272.pdf>
    Local Reference: \texttt{refs/9201272v1.pdf}
    "Define the modulus mod(C) of such a cylinder to be the ratio $\Delta$y/$\Delta$x of height to circumference."
    "Corollary B.4. The modulus of a cylinder is a well defined conformal invariant."
    "It follows that the modulus of an annulus A can be defined as the modulus
    of any conformally isomorphic cylinder." -/
opaque raw\_modulus (A : Set $\mathbb{C}$) : $\mathbb{R}$

/-- The modulus of an annulus.
    Defined to be 0 for the empty set, and the \texttt{raw\_modulus} otherwise. -/
noncomputable def modulus (A : Set $\mathbb{C}$) : $\mathbb{R}$ :=
  if A = $\emptyset$ then 0 else raw\_modulus A

/-- The modulus of the empty set is 0.
    See: [Milnor, Dynamics in One Complex Variable, Appendix B] <https://arxiv.org/pdf/math/9201272.pdf>
    Local Reference: \texttt{refs/9201272v1.pdf}
    "By definition an infinite cylinder, that is a cylinder of infinite height, has modulus zero."
    (Note: Typically empty sets or degenerate annuli are treated as limiting
    cases or specific values like 0 or infinity depending on convention; Milnor
    assigns 0 to infinite cylinders in some contexts or infinite modulus to
    complements of points. Here we assume standard convention for empty
    annulus).

\textit{(See code: modulus\_empty)}

\subsection*{modulus\_nonneg\_ax}\label{modulus_nonneg_ax}
Modulus is non-negative.
    This follows from the definition of modulus as a conformal invariant.
    See: [Milnor, Dynamics in One Complex Variable] <https://arxiv.org/pdf/math/9201272.pdf>
    Local Reference: \texttt{refs/9201272v1.pdf}
    "Define the modulus mod(C) of such a cylinder to be the ratio $\Delta$y/$\Delta$x of
    height to circumference." (Ratio of positive lengths is positive).

\textit{(See code: modulus\_nonneg\_ax)}

\subsection*{groetzsch\_inequality}\label{groetzsch_inequality}
Grötzsch's Inequality: Superadditivity of modulus for disjoint essential annuli.
    See: [Milnor, Dynamics in One Complex Variable, Corollary B.5] <https://arxiv.org/pdf/math/9201272.pdf>
    Local Reference: \texttt{refs/9201272v1.pdf}
    "Corollary B.5 (Grötzsch Inequality). Suppose that A' $\subset$ A and A'' $\subset$ A are
    two disjoint annuli, each essentailly embedded in A. Then mod(A') + mod(A'')
    $\le$ mod(A)."

\textit{(See code: groetzsch\_inequality)}

\subsection*{modulus\_summable\_of\_nontrivial\_intersection}\label{modulus_summable_of_nontrivial_intersection}
Grötzsch's Inequality implies summability if the intersection is non-trivial.
    Proof idea: We construct a sequence of disjoint annuli \texttt{A\_n = P\_n \textbackslash{} P\_\{n+1\}}.
    By the contrapositive of Grötzsch's criterion (or directly by inequality), if the intersection
    is non-trivial (contains more than just a point), there is a core \texttt{K} inside all \texttt{P\_n}.
    The disjoint annuli are all nested around \texttt{K}. Grötzsch's inequality implies their moduli
    sum up to at most the modulus of the container \texttt{P\_0 \textbackslash{} K}, which is finite.
    Thus the sum converges.

\begin{figure}[ht]
    \centering
    \includegraphics[width=0.9\textwidth]{images/groetzsch_viz.png}
    \caption{Visualization of Grötzsch's Criterion. Left: If the nested annuli $A_n$ enclose a non-trivial set $K$ (red), their total modulus must be finite (bounded by the modulus of $P_0 \setminus K$). Right: If the sum of moduli diverges to infinity, the intersection must shrink to a single point.}
    \label{fig:groetzsch}
\end{figure}

\textit{(See code: modulus\_summable\_of\_nontrivial\_intersection)}

\subsection*{groetzsch\_criterion}\label{groetzsch_criterion}
Grötzsch's Criterion: Divergence of moduli implies point intersection.
    See: [Milnor, Dynamics in One Complex Variable, Corollary B.7]
    Local Reference: \texttt{refs/9201272v1.pdf}
    "Corollary B.7. Suppose that K $\subset$ U as described above. Then K reduces to a single point if and only if the annulus A = U rK has infinite modulus."

    Proof idea: We argue by contrapositive. If the intersection is non-trivial (contains more than just \texttt{\{0\}}),
    then \texttt{modulus\_summable\_of\_nontrivial\_intersection} implies the sum of moduli converges.
    This contradicts the hypothesis that the sum diverges.
    Therefore, the intersection must be trivial (equal to \texttt{\{0\}}).

\textit{(See code: groetzsch\_criterion)}

