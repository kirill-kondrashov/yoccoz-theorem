\section{Grötzsch's Inequality and Criterion}
\label{sec:Groetzsch}

This section describes the verification of the file \texttt{Mlc/Quadratic/Complex/Groetzsch.lean}. It formalizes the conformal modulus of an annulus and establishes the key analytical tool for the proof: Grötzsch's Inequality and its associated convergence criterion.

\subsection{Modulus Definition}

We begin by defining the modulus of an annulus. Following Milnor (Appendix B), the modulus is a conformal invariant.

\begin{lstlisting}[language=Lean]
opaque raw_modulus (A : Set ℂ) : ℝ

/-- The modulus of an annulus.
    Defined to be 0 for the empty set, and the `raw_modulus` otherwise. -/
noncomputable def modulus (A : Set ℂ) : ℝ :=
  if A = ∅ then 0 else raw_modulus A
\end{lstlisting}

\noindent
\textbf{Detailed Explanation:}
\begin{itemize}
    \item \texttt{raw\_modulus (A : Set ℂ)}: This is an opaque definition representing the conformal modulus. In a full formalization from first principles, this would be defined via extremal length or conformal mapping to a standard cylinder. Here, we treat it axiomatically as a real-valued function.
    \item \texttt{modulus}: A wrapper around \texttt{raw\_modulus} that explicitly handles the empty set case, ensuring \texttt{modulus ∅ = 0}.
\end{itemize}

We establish basic properties of the modulus:

\begin{lstlisting}[language=Lean]
theorem modulus_empty : modulus ∅ = 0 := by
  simp [modulus]

axiom modulus_nonneg_ax (A : Set ℂ) : 0 ≤ raw_modulus A

theorem modulus_nonneg (A : Set ℂ) : 0 ≤ modulus A := by
  unfold modulus
  split_ifs
  · exact le_refl 0
  · exact modulus_nonneg_ax A
\end{lstlisting}

\noindent
\textbf{Detailed Explanation:}
\begin{itemize}
    \item \texttt{modulus\_empty}: Trivially true by the definition of \texttt{modulus}.
    \item \texttt{modulus\_nonneg}: Ensures the modulus is always non-negative. This is split into cases: 0 for the empty set, and non-negative by axiom for non-empty sets.
\end{itemize}

\begin{proof}
If $A=\varnothing$ then $\text{modulus}(A)=0$, so $0\le \text{modulus}(A)$. Otherwise $\text{modulus}(A)=\text{raw\_modulus}(A)$ and by the axiom $0\le \text{raw\_modulus}(A)$, hence $0\le \text{modulus}(A)$.
\end{proof}

\subsection{Grötzsch's Inequality}

The fundamental property of the modulus used here is superadditivity for disjoint nested annuli. This is known as Grötzsch's Inequality.

\begin{lstlisting}[language=Lean]
axiom groetzsch_inequality {A B S : Set ℂ} (h_disj : Disjoint A B) (h_sub : A ∪ B ⊆ S) :
    modulus A + modulus B ≤ modulus S
\end{lstlisting}

\noindent
\textbf{Detailed Explanation:}
\begin{itemize}
    \item \texttt{h\_disj}: Requires the sub-annuli $A$ and $B$ to be disjoint.
    \item \texttt{h\_sub}: Requires both $A$ and $B$ to be contained in the "parent" annulus $S$.
    \item The conclusion \texttt{modulus A + modulus B ≤ modulus S} states that the sum of moduli of disjoint sub-annuli cannot exceed the modulus of the container. This captures the intuition that modulus measures the "thickness" of the annulus.
\end{itemize}

\subsection{Helper Lemma: Nested Subsets}

A utility lemma is needed to handle nested sequences of sets.

\begin{lstlisting}[language=Lean]
lemma subset_of_le_nested {P : ℕ → Set ℂ} (h_nested : ∀ n, P (n + 1) ⊆ P n)
    {i j : ℕ} (hij : i ≤ j) : P j ⊆ P i := by
  have h_diff : ∃ k, j = i + k := Nat.exists_eq_add_of_le hij
  obtain ⟨k, rfl⟩ := h_diff
  clear hij
  induction k with
  | zero => exact subset_refl _
  | succ m ih =>
    rw [Nat.add_succ]
    apply subset_trans (h_nested (i + m)) ih
\end{lstlisting}

\noindent
\textbf{Detailed Explanation:}
\begin{itemize}
    \item This lemma generalizes the step-by-step nesting \texttt{P (n+1) ⊆ P n} to arbitrary indices \texttt{i ≤ j}.
    \item The proof proceeds by induction on the difference $k = j - i$.
    \item \texttt{Nat.exists\_eq\_add\_of\_le}: Extracts the difference $k$.
    \item Base case (\texttt{zero}): $P_i \subseteq P_i$ is reflexive.
    \item Inductive step (\texttt{succ}): Uses transitivity: $P_{i+m+1} \subseteq P_{i+m} \subseteq P_i$.
\end{itemize}

\begin{proof}
Write $j=i+k$ and proceed by induction on $k$. The base case $k=0$ is trivial since $P_i\subseteq P_i$. For the inductive step, assume $P_{i+m+1}\subseteq P_i$ when $j=i+m$; then using $P_{i+m+1}\subseteq P_{i+m}$ and the transitive inclusion $P_{i+m}\subseteq P_i$ yields $P_{i+m+1}\subseteq P_i$, completing the induction.
\end{proof}

\subsection{Summability from Non-trivial Intersection}

The core logical step for the criterion is proving that if the intersection of nested sets is "fat" (non-trivial), the sum of moduli of the difference annuli must converge.

\begin{lstlisting}[language=Lean]
theorem modulus_summable_of_nontrivial_intersection {P : ℕ → Set ℂ}
    (h_nested : ∀ n, P (n + 1) ⊆ P n)
    (_h_conn : ∀ n, IsConnected (P n))
    (_h_nontriv : Set.Nontrivial (⋂ n, P n)) :
    Summable (fun n => modulus (P n \ P (n + 1))) := by
  let A := fun n => P n \ P (n + 1)
  have h_disj : ∀ i j, i < j → Disjoint (A i) (A j) := by
    intro i j hij
    rw [Set.disjoint_left]
    intro z hi hj
    simp [A] at hi hj
    have h_sub : P j ⊆ P (i + 1) := subset_of_le_nested h_nested hij
    have z_in_P_j := hj.1
    have z_in_P_i_1 := h_sub z_in_P_j
    have z_not_in_P_i_1 := hi.2
    contradiction

  have h_union_sub : ∀ N, (⋃ n ∈ Finset.range N, A n) ⊆ P 0 \ P N := by
    intro N
    rw [Set.subset_def]
    intro z hz
    simp at hz
    obtain ⟨n, hn_lt, hn_z⟩ := hz
    simp [A] at hn_z
    constructor
    · -- z ∈ P 0
      have h_sub : P n ⊆ P 0 := subset_of_le_nested h_nested (Nat.zero_le n)
      exact h_sub hn_z.1
    · -- z ∉ P N
      intro h_in_N
      have h_sub : P N ⊆ P (n + 1) := subset_of_le_nested h_nested hn_lt
      apply hn_z.2
      apply h_sub h_in_N

  -- Monotonicity lemma
  have modulus_mono : ∀ {U V : Set ℂ}, U ⊆ V → modulus U ≤ modulus V := by
    intro U V h_sub
    have h_union : U ∪ ∅ ⊆ V := by simp [h_sub]
    have h_disj_empty : Disjoint U ∅ := disjoint_empty U
    have h_ineq := groetzsch_inequality h_disj_empty h_union
    rw [modulus_empty, add_zero] at h_ineq
    exact h_ineq

  -- Bounded partial sums
  have h_bounded : ∀ N, Finset.sum (Finset.range N) (fun n => modulus (A n)) ≤ modulus (P 0 \ (⋂ n, P n)) := by
    intro N
    -- First show sum ≤ modulus (P 0 \ P N)
    have h_sum_le : Finset.sum (Finset.range N) (fun n => modulus (A n)) ≤ modulus (P 0 \ P N) := by
      induction N with
      | zero =>
        simp
        rw [modulus_empty]
      | succ k ih =>
        rw [Finset.sum_range_succ]
        have h_split : P 0 \ P (k + 1) = (P 0 \ P k) ∪ (P k \ P (k + 1)) := by
          ext z
          simp
          constructor
          · intro h
            by_cases hk : z ∈ P k
            · right; exact ⟨hk, h.2⟩
            · left; exact ⟨h.1, hk⟩
          · intro h
            cases h with
            | inl h => exact ⟨h.1, fun h_in => h.2 (h_nested k h_in)⟩
            | inr h =>
                have h_sub : P k ⊆ P 0 := subset_of_le_nested h_nested (Nat.zero_le k)
                exact ⟨h_sub h.1, h.2⟩

        have h_disj_split : Disjoint (P 0 \ P k) (P k \ P (k + 1)) := by
          rw [Set.disjoint_left]
          intro z h1 h2
          have h_in_Pk := h2.1
          have h_not_in_Pk := h1.2
          contradiction

        have h_ineq := groetzsch_inequality h_disj_split (subset_of_eq h_split.symm)
        apply le_trans (add_le_add ih (le_refl (modulus (A k))))
        exact h_ineq

    apply le_trans h_sum_le
    apply modulus_mono
    apply diff_subset_diff_right
    apply sInter_subset_of_mem
    simp

  apply summable_of_sum_range_le (fun n => modulus_nonneg _) h_bounded
\end{lstlisting}

\noindent
\textbf{Detailed Explanation:}
\begin{itemize}
    \item \textbf{Goal}: Prove that $\sum \text{mod}(P_n \setminus P_{n+1})$ converges.
    \item \texttt{A := fun n => P n \textbackslash P (n + 1)}: Defines the sequence of annuli $A_n$.
    \item \texttt{h\_disj}: Proves that $A_i$ and $A_j$ are disjoint for $i < j$. This relies on the nested property: $A_j \subseteq P_j \subseteq P_{i+1}$, while $A_i$ is disjoint from $P_{i+1}$.
    \item \texttt{h\_union\_sub}: Shows that the union of the first $N$ annuli is contained in $P_0 \setminus P_N$.
    \item \texttt{modulus\_mono}: Proves monotonicity of modulus ($U \subseteq V \implies \text{mod}(U) \le \text{mod}(V)$). This is derived from Grötzsch's inequality by taking the second set to be empty.
    \item \texttt{h\_bounded}: Proves the partial sums are bounded.
        \begin{itemize}
            \item \texttt{h\_sum\_le}: Shows $\sum_{n=0}^{N-1} \text{mod}(A_n) \le \text{mod}(P_0 \setminus P_N)$ by induction.
            \item The inductive step uses \texttt{h\_split} to decompose $P_0 \setminus P_{k+1}$ into $(P_0 \setminus P_k) \cup (P_k \setminus P_{k+1})$.
            \item Grötzsch's inequality is then applied to this disjoint union.
            \item Finally, we bound $\text{mod}(P_0 \setminus P_N)$ by $\text{mod}(P_0 \setminus \bigcap P_n)$ using monotonicity, noting that $P_N \supseteq \bigcap P_n$ implies $P_0 \setminus P_N \subseteq P_0 \setminus \bigcap P_n$.
        \end{itemize}
    \item \texttt{summable\_of\_sum\_range\_le}: Concludes convergence since the partial sums are bounded and terms are non-negative.
\end{itemize}

\begin{proof}
The central step is the induction which shows for every N:
\[
\sum_{n=0}^{N-1}\mathrm{mod}(A_n)\le\mathrm{mod}(P_0\setminus P_N).
\]
For $N=0$ the claim is trivial. For the inductive step decompose
\[
P_0\setminus P_{k+1}=(P_0\setminus P_k)\cup(P_k\setminus P_{k+1}),
\]
and note the union is disjoint. Applying Grötzsch's inequality to this decomposition yields
\[
\mathrm{mod}(P_0\setminus P_k)+\mathrm{mod}(P_k\setminus P_{k+1})\le\mathrm{mod}(P_0\setminus P_{k+1}).
\]
By the inductive hypothesis the left-hand side bounds the partial sum up to index $k$, hence the partial sum up to $k+1$ is bounded by the modulus of the outer difference. Iterating and passing to the intersection gives a uniform bound for all partial sums, and since the terms are non-negative the series is summable.
\end{proof}

\subsection{Grötzsch's Criterion}

Finally, we prove the main criterion: if the sum of moduli diverges, the intersection must be a single point (assuming it contains 0).

\begin{lstlisting}[language=Lean]
theorem groetzsch_criterion {P : ℕ → Set ℂ}
    (h_nested : ∀ n, P (n + 1) ⊆ P n)
    (h_zero : ∀ n, 0 ∈ P n)
    (h_conn : ∀ n, IsConnected (P n))
    (h_div : ¬ Summable (fun n => modulus (P n \ P (n + 1)))) :
    (⋂ n, P n) = {0} := by
  by_contra h_neq
  have h_nontriv : Set.Nontrivial (⋂ n, P n) := by
    have h_0 : 0 ∈ ⋂ n, P n := Set.mem_iInter.mpr h_zero
    rw [Set.nontrivial_iff_exists_ne h_0]
    by_contra h_all_eq
    apply h_neq
    ext z
    constructor
    · intro hz
      by_cases h_z_eq : z = 0
      · rw [h_z_eq]; exact Set.mem_singleton 0
      · push_neg at h_all_eq
        specialize h_all_eq z hz
        contradiction
    · intro hz
      rw [Set.mem_singleton_iff] at hz
      rw [hz]
      exact h_0
  have h_sum := modulus_summable_of_nontrivial_intersection h_nested h_conn h_nontriv
  contradiction
\end{lstlisting}

\noindent
\textbf{Detailed Explanation:}
\begin{itemize}
    \item \textbf{Hypotheses}:
        \begin{itemize}
            \item \texttt{h\_nested}: Sets are nested.
            \item \texttt{h\_zero}: All sets contain 0.
            \item \texttt{h\_div}: The sum of moduli diverges.
        \end{itemize}
    \item \textbf{Proof Strategy}: Proof by contradiction (\texttt{by\_contra}).
    \item \texttt{h\_neq}: Assume the intersection is not equal to $\{0\}$.
    \item \texttt{h\_nontriv}: We deduce that the intersection is "non-trivial" (contains at least two points). Since we know $0$ is in the intersection, and the intersection is not just $\{0\}$, there must be some $z \ne 0$.
    \item \texttt{h\_sum}: We apply \texttt{modulus\_summable\_of\_nontrivial\_intersection}. This theorem tells us that if the intersection is non-trivial, the sum of moduli \emph{must} converge.
    \item \textbf{Contradiction}: We have derived that the sum is summable (\texttt{h\_sum}), which contradicts the hypothesis \texttt{h\_div} (that the sum is not summable).
    \item Therefore, the assumption \texttt{h\_neq} must be false, and the intersection is indeed exactly $\{0\}$.
\end{itemize}

\begin{proof}
Assume the intersection contains a point $z\neq0$; then it is nontrivial and by the previous theorem the series of moduli is summable. This contradicts the hypothesis that the series diverges, so the intersection must equal $\{0\}$.
\end{proof}
