\section{Properties of the Green's Function}
\label{sec:GreenLemmas}

This section details the convergence properties of the sequence defining the Green's function.

\subsection*{Potential Sequence}
The Green's function is constructed as the limit of the sequence:
\[ G_n(z) = \frac{1}{2^n} \log \max(1, |f_c^n(z)|). \]
This sequence captures the exponential growth rate of the orbit norm.

\subsection*{Convergence}
\textbf{Theorem} (\texttt{potential\_seq\_converges}).
\textit{For any $c, z \in \mathbb{C}$, the sequence $G_n(z)$ converges to a finite limit $G_c(z)$.}

\begin{proof}[Proof Idea]
The proof splits into two cases:
\begin{itemize}
    \item \textbf{Bounded Orbits ($z \in K(c)$)}: The orbit $|f_c^n(z)|$ remains bounded by some $M$. Thus,
    \[ 0 \le G_n(z) \le \frac{1}{2^n} \log(\max(1, M)) \to 0. \]
    So $G_c(z) = 0$.
    \item \textbf{Escaping Orbits ($z \notin K(c)$)}: Once the orbit exceeds the escape bound $R(c)$, the growth is approximately squaring ($|z_{n+1}| \approx |z_n|^2$). We show the sequence is Cauchy by bounding the difference:
    \[ |G_{n+1}(z) - G_n(z)| \approx \frac{1}{2^{n+1}} \left| \log \frac{|f_c(z_n)|}{|z_n|^2} \right| \le \frac{C}{2^n |z_n|^2}. \]
    Since $|z_n|$ grows exponentially, the sum of differences converges rapidly.
\end{itemize}

\begin{figure}[ht]
    \centering
    \includegraphics[width=1.0\textwidth]{images/convergence_plots.png}
    \caption{Visualization of the convergence of the potential sequence $G_n(z)$. Left: The sequence values stabilize rapidly for escaping points (orange, blue) and stay zero for bounded points (green). Right: The difference $|G_{n+1}-G_n|$ decays doubly exponentially for escaping points, illustrating the Cauchy property.}
    \label{fig:convergence}
\end{figure}

\end{proof}

\subsection*{Limit Definition}
\textbf{Lemma} (\texttt{green\_function\_eq\_lim}).
\textit{The function \texttt{green\_function} coincides with the limit of the potential sequence:}
\[ G_c(z) = \lim_{n \to \infty} G_n(z). \]
For a visualization of the limit function $G_c(z)$ and its geometry, see Figure \ref{fig:green_function}.

\subsection*{Escape Bound}
The convergence proof relies on an explicit escape bound slightly larger than $R(c)$ to ensure stable logarithmic behavior.
\textbf{Definition} (\texttt{escape\_bound}).
\[ B(c) = \max\left(R(c), \sqrt{2|c| + 1}\right). \]
This bound ensures that once $|z| > B(c)$, the term $|c|/|z|^2$ is small enough (specifically $\le 1/2$) to use Taylor expansion estimates for the logarithm.
