\section{Yoccoz}
\textit{Source: \texttt{Yoccoz.lean}}

This module contains the formal definitions and proofs.
\subsection*{NonRenormalizable}\label{NonRenormalizable}
Non-renormalizable parameters.
    For the purpose of this plan, we define non-renormalizable parameters
    as those for which the Yoccoz puzzle moduli diverge.
    The deep work is then in the dichotomy axiom.

\textit{(See code: NonRenormalizable)}

\subsection*{non\_renormalizable\_moduli\_diverge}\label{non_renormalizable_moduli_diverge}
Non-renormalizable parameters have divergent moduli.

\textit{(See code: non\_renormalizable\_moduli\_diverge)}

\subsection*{yoccoz\_theorem}\label{yoccoz_theorem}
Yoccoz's Theorem: Divergence of moduli implies point intersection.
    Proof idea:
    *   If \texttt{c $\in$ M}: We apply **Grötzsch's criterion** to the nested sequence of dynamical puzzle pieces.
        These pieces contain 0 and are connected. The divergence of the moduli of the annuli between
        them forces the intersection of the pieces to be a single point \texttt{\{0\}}.
    *   If \texttt{c $\notin$ M}: The orbit of 0 escapes. For large enough \texttt{n}, the potential level \texttt{1/2\textasciicircum{}n}
        is smaller than \texttt{G(0)}, so \texttt{0} is no longer in the puzzle piece (which is defined by \texttt{G(z) < 1/2\textasciicircum{}n}).
        Thus, the puzzle pieces eventually become empty. This would imply the sum of moduli is finite
        (sum of zeros), contradicting the divergence hypothesis. Thus, this case is impossible under
        the assumption of divergence.

\textit{(See code: yoccoz\_theorem)}

