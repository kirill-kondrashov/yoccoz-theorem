\section{Yoccoz's Theorem}
\label{sec:Yoccoz}

This section connects the analytical tools (Grötzsch's inequality) with the combinatorial puzzle structure to prove the main result: for non-renormalizable parameters, the dynamical puzzle pieces shrink to a point.

\subsection{Non-Renormalizability}

The concept of renormalizability is central to the classification of quadratic dynamics. For the purpose of this formalization, we adopt a combinatorial definition based on the puzzle moduli.

\begin{definition}[Non-Renormalizable Parameter]
A parameter $c \in \mathbb{C}$ is called \textit{non-renormalizable} if the sum of the moduli of the puzzle annuli diverges:
\[ \sum_{n=0}^{\infty} \operatorname{mod}(A_n) = \infty, \]
where $A_n = P_n(0) \setminus P_{n+1}(0)$ is the puzzle annulus at depth $n$.
\end{definition}

In the full theory, this condition is derived from the combinatorial "tableau" of the puzzle. Here, we take it as the defining property (Definition \texttt{NonRenormalizable}).

\subsection{Yoccoz's Theorem}

The main theorem asserts that this combinatorial condition forces local connectivity at the critical point.

\begin{theorem}[Yoccoz's Theorem]
Let $c \in \mathbb{C}$. If the parameter is non-renormalizable (i.e., the moduli sum diverges), then the intersection of all dynamical puzzle pieces containing $0$ consists of the single point $0$:
\[ \bigcap_{n=0}^{\infty} P_n(0) = \{0\}. \]
\end{theorem}

\begin{proof}[Proof Strategy]
The proof considers two cases based on whether $c$ is in the Mandelbrot set.

\textbf{Case 1: $c \in \mathcal{M}$.}
If $c$ is in the Mandelbrot set, the critical point $0$ never escapes. The puzzle pieces $P_n(0)$ form a nested sequence of connected open sets containing $0$.
Since the parameter is non-renormalizable, we have by hypothesis:
\[ \sum \operatorname{mod}(A_n) = \infty. \]
By \textbf{Grötzsch's Criterion} (Theorem \ref{thm:groetzsch_criterion}), this divergence immediately implies that the intersection is trivial:
\[ \bigcap P_n(0) = \{0\}. \]
This is the core of the argument.

\textbf{Case 2: $c \notin \mathcal{M}$.}
If $c$ is outside the Mandelbrot set, the orbit of $0$ escapes to infinity. As shown in Lemma \texttt{dynamical\_puzzle\_piece\_empty\_of\_large\_n}, for sufficiently large $n$, $P_n(0)$ becomes empty (or does not contain $0$).
If the sequence becomes empty, the moduli of the difference annuli would eventually be $0$ (since $\operatorname{mod}(\emptyset) = 0$).
This would imply the sum is finite: $\sum \operatorname{mod}(A_n) < \infty$.
But this contradicts the assumption that $c$ is non-renormalizable (divergent sum). Thus, this case cannot occur for a non-renormalizable parameter.
\end{proof}

\begin{figure}[ht]
    \centering
    \includegraphics[width=1.0\textwidth]{images/yoccoz_concept.png}
    \caption{Visualization of Yoccoz's Theorem. Left: Global view of the Basilica Julia set ($c=-1$), showing the location of the critical point $0$ deep inside the filled set. Right: Conceptual model of the local dynamics near $0$. The theorem states that if the sum of moduli of the nested annuli $A_n$ diverges, the puzzle pieces $P_n$ must shrink to the single point $\{0\}$.}
    \label{fig:yoccoz_concept}
\end{figure}

\subsection{Implications}
This result is a key step towards the MLC conjecture. Since $P_n(0)$ shrinks to $\{0\}$, the fiber of the puzzle at the critical point is trivial. Through further arguments (relating the dynamical plane to the parameter plane), this implies local connectivity of the Mandelbrot set at $c$.


