\section{Yoccoz's Theorem}
\label{sec:Yoccoz}

This section describes the verification of the file \texttt{Mlc/Yoccoz.lean}. It proves the central result for non-renormalizable parameters: the divergence of puzzle moduli implies the shrinking of puzzle pieces.

\subsection{Non-Renormalizability}

We define non-renormalizability via the combinatorial condition of divergent moduli.

\begin{lstlisting}[language=Lean]
/-- Non-renormalizable parameters.
    For the purpose of this plan, we define non-renormalizable parameters
    as those for which the Yoccoz puzzle moduli diverge.
    The deep work is then in the dichotomy axiom. -/
def NonRenormalizable (c : ℂ) : Prop :=
    ¬ Summable (fun n => modulus (PuzzleAnnulus c n))

/-- Non-renormalizable parameters have divergent moduli. -/
theorem non_renormalizable_moduli_diverge (c : ℂ) (h : NonRenormalizable c) :
    ¬ (Summable fun n => modulus (PuzzleAnnulus c n)) := h
\end{lstlisting}

\noindent
\textbf{Simple proof idea:}
\begin{itemize}
    \item The definition \texttt{NonRenormalizable c} is exactly the proposition that the sequence of annulus moduli is not summable.
    \item The Lean theorem \texttt{non\_renormalizable\_moduli\_diverge} is therefore a one-line wrapper that returns the hypothesis: given \texttt{h : NonRenormalizable c}, the conclusion follows immediately by returning \texttt{h}.
\end{itemize}

\begin{proof}
Immediate from the definition: the hypothesis \texttt{h} is the required conclusion, so the proof is simply \texttt{exact h}.
\end{proof}

\noindent
\textbf{Detailed Explanation:}
\begin{itemize}
    \item \texttt{NonRenormalizable c}: Defined as the negation of summability for the moduli of puzzle annuli.
    \item This definition serves as the hypothesis for applying Grötzsch's criterion.
\end{itemize}

\subsection{Yoccoz's Theorem}

The main theorem establishes that for these parameters, the dynamical puzzle pieces around the critical point shrink to exactly $\{0\}$.

\begin{lstlisting}[language=Lean]
theorem yoccoz_theorem (c : ℂ) :
    ¬ (Summable fun n => modulus (PuzzleAnnulus c n)) →
    (⋂ n, DynamicalPuzzlePiece c n 0) = {0} := by
  intro h_div
  by_cases hc : c ∈ MandelbrotSet
  · apply groetzsch_criterion
    · intro n
      apply dynamical_puzzle_piece_nested
    · intro n
      apply mem_dynamical_puzzle_piece_self c hc
    · intro n
      have h_ne : (DynamicalPuzzlePiece c n 0).Nonempty := ⟨0, mem_dynamical_puzzle_piece_self c hc n⟩
      rw [DynamicalPuzzlePiece] at h_ne ⊢
      exact ⟨h_ne, isPreconnected_connectedComponentIn⟩
    · exact h_div
  · exfalso
    apply h_div
    rcases dynamical_puzzle_piece_empty_of_large_n c hc with ⟨N, hN⟩
    apply summable_of_finite_support
    have : (Function.support fun n ↦ modulus (PuzzleAnnulus c n)) ⊆ Iio N := by
      intro n hn
      rw [Function.mem_support, ne_eq] at hn
      by_contra h_ge
      simp at h_ge
      have : modulus (PuzzleAnnulus c n) = 0 := by
        rw [PuzzleAnnulus]
        have h_empty : DynamicalPuzzlePiece c n 0 = ∅ := by
          ext x
          simp
          intro hx
          have h0 : 0 ∈ DynamicalPuzzlePiece c n 0 := by
            rw [DynamicalPuzzlePiece] at hx ⊢
            apply mem_connectedComponentIn
            exact connectedComponentIn_nonempty_iff.1 ⟨x, hx⟩
          exact hN n h_ge h0
        rw [h_empty]
        simp
        exact modulus_empty
      contradiction
    exact Set.Finite.subset (Set.finite_Iio N) this
\end{lstlisting}

\noindent
\textbf{Detailed Explanation:}
\begin{itemize}
    \item \textbf{Goal}: Prove $\bigcap P_n(0) = \{0\}$ assuming divergence of moduli.
    \item \textbf{Case 1: $c \in \mathcal{M}$}:
        \begin{itemize}
            \item We apply \texttt{groetzsch\_criterion}.
            \item \textbf{Hypotheses Check}:
                \begin{itemize}
                    \item \texttt{nested}: Verified by \texttt{dynamical\_puzzle\_piece\_nested}.
                    \item \texttt{contains 0}: Verified by \texttt{mem\_dynamical\_puzzle\_piece\_self}.
                    \item \texttt{connected}: Verified by construction (connected components).
                    \item \texttt{divergence}: Assumed by hypothesis \texttt{h\_div}.
                \end{itemize}
            \item The criterion then immediately yields the result.
        \end{itemize}
    \item \textbf{Case 2: $c \notin \mathcal{M}$}:
        \begin{itemize}
            \item We derive a contradiction.
            \item If $c \notin \mathcal{M}$, then the puzzle pieces eventually become empty (\texttt{dynamical\_\allowbreak puzzle\_\allowbreak piece\_\allowbreak empty\_\allowbreak of\_\allowbreak large\_n}).
            \item This means the annuli become empty for large $n$, so their moduli are 0.
            \item A sequence of non-negative terms that is eventually 0 has a finite sum (\texttt{summable\_\allowbreak of\_\allowbreak finite\_\allowbreak support}).
            \item This contradicts the divergence hypothesis \texttt{h\_div}.
            \item Thus, non-renormalizable parameters (by this definition) must lie in the Mandelbrot set.
        \end{itemize}
\end{itemize}

\begin{proof}
Simplified proof: If $c\in\mathcal{M}$, apply Grötzsch's criterion to the nested, connected dynamical pieces that all contain $0$; divergence of the annulus moduli forces their intersection to be the single point $\{0\}$. If $c\notin\mathcal{M}$ the pieces are eventually empty, so the moduli vanish and their sum is finite, contradicting the divergence hypothesis. Therefore only the first case can occur and the intersection equals $\{0\}$.
\end{proof}
