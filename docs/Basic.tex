\section{Basic Definitions}
\label{sec:Basic}

In this section, we establish the fundamental definitions for the dynamics of quadratic polynomials, following the standard notation (e.g., Lyubich).

\subsection{The Quadratic Family}
We consider the family of quadratic polynomials parametrized by $c \in \mathbb{C}$:
\[
f_c(z) = z^2 + c
\]

\begin{figure}[ht]
    \centering
    \includegraphics[width=0.8\textwidth]{images/basic_mapping.png}
    \caption{The Quadratic Map $z \mapsto z^2 + c$. The grid lines in the domain (blue) are mapped to parabolic curves in the range (red), illustrating the "folding" nature of the map.}
    \label{fig:basic_mapping}
\end{figure}

The forward orbit of a point $z_0 \in \mathbb{C}$ under $f_c$ is the sequence defined by:
\[
\text{orbit}_c(z_0, n) = f_c^{\circ n}(z_0)
\]
where $f_c^{\circ n}$ denotes the $n$-th iteration of $f_c$.

\subsection{Julia and Mandelbrot Sets}
A point $z \in \mathbb{C}$ has a \textbf{bounded orbit} if there exists $M \in \mathbb{R}$ such that $|f_c^{\circ n}(z)| \le M$ for all $n$.

The \textbf{Filled Julia Set}, denoted $K(c)$, is the set of points with bounded orbits:
\[
K(c) = \{ z \in \mathbb{C} \mid \text{orbit of } z \text{ is bounded} \}
\]
The \textbf{Julia Set} $J(c)$ is defined as the topological boundary of $K(c)$:
\[
J(c) = \partial K(c)
\]

\begin{figure}[ht]
    \centering
    \includegraphics[width=0.5\textwidth]{images/basic_julia.png}
    \caption{The Filled Julia Set $K(c)$ (black) and its boundary, the Julia Set $J(c)$, for $c \approx -0.73 + 0.19i$. Points in black have bounded orbits; colored regions represent the basin of infinity with potential levels.}
    \label{fig:basic_julia}
\end{figure}

The \textbf{Mandelbrot Set} $\mathcal{M}$ is the set of parameters $c$ for which the critical point $0$ has a bounded orbit:
\[
\mathcal{M} = \{ c \in \mathbb{C} \mid 0 \in K(c) \}
\]

\begin{figure}[ht]
    \centering
    \includegraphics[width=0.6\textwidth]{images/basic_mandelbrot.png}
    \caption{The Mandelbrot Set $\mathcal{M}$ (black). It represents the set of parameters $c$ for which the Julia set $K(c)$ is connected.}
    \label{fig:basic_mandelbrot}
\end{figure}

\subsection{Escape Radius}
We define an escape radius $R(c)$ depending on the parameter $c$:
\[
R(c) = \max(2, 1 + |c|)
\]
It is well-known that if the modulus of an iterate exceeds $R(c)$, the orbit escapes to infinity.

\subsection{Fundamental Axioms}
We assume two deep results from the theory of complex dynamics as axioms for this formalization:

\begin{itemize}
    \item \textbf{Connectivity of $\mathcal{M}$}: The Mandelbrot set $\mathcal{M}$ is connected (Douady and Hubbard, 1984).
    \item \textbf{Connectivity of $K(c)$}: For any $c \in \mathcal{M}$, the filled Julia set $K(c)$ is connected (Douady and Hubbard, 1984).
\end{itemize}

\begin{figure}[ht]
    \centering
    \includegraphics[width=0.8\textwidth]{images/basic_connectivity.png}
    \caption{Connectivity Dichotomy. Left: For $c \in \mathcal{M}$, $K(c)$ is connected. Right: For $c \notin \mathcal{M}$, $K(c)$ is a Cantor set (disconnected dust).}
    \label{fig:basic_connectivity}
\end{figure}
