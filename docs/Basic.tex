\section{Basic Definitions}
\label{sec:Basic}

This section describes the verification of the file \texttt{Mlc/Quadratic/Complex/Basic.lean}. It establishes the fundamental definitions for the dynamics of quadratic polynomials, utilizing Lyubich's notation.

\subsection{Quadratic Family}

We set up the quadratic family $f_c(z) = z^2 + c$, its iterates, the filled Julia set $K(c)$, and the Julia set $J(c) = \partial K(c)$.

\subsubsection{Definitions}

We begin by defining the quadratic polynomial and its orbit.

\begin{lstlisting}[language=Lean]
/-- The quadratic polynomial `f_c(z) = z^2 + c`. -/
def fc (c : $\mathbb{C}$) : $\mathbb{C}$ $\to$ $\mathbb{C}$ := fun z => z^2 + c

/-- The forward orbit of `z0` under `f_c`. -/
def orbit (c : $\mathbb{C}$) (z0 : $\mathbb{C}$) : $\mathbb{N}$ $\to$ $\mathbb{C}$ := fun n => (Nat.iterate (fc c) n) z0
\end{lstlisting}

\textbf{Explanation:}
\begin{itemize}
    \item \texttt{fc}: Defines the map $z \mapsto z^2 + c$.
    \item \texttt{orbit}: Defines the sequence $z_n = f_c^n(z_0)$ using \texttt{Nat.iterate}.
\end{itemize}

\subsubsection{Orbit Lemmas}

\begin{lstlisting}[language=Lean]
@[simp] lemma orbit_zero (c z : $\mathbb{C}$) : orbit c z 0 = z := rfl

@[simp] lemma orbit_succ (c z : $\mathbb{C}$) (n : $\mathbb{N}$) :
    orbit c z (n+1) = fc c (orbit c z n) := by
  simpa [orbit, Function.comp] using
    congrArg (fun g => g z) (Function.iterate_succ' (fc c) n)
\end{lstlisting}

\textbf{Explanation:}
These lemmas provide the recursive structure of the orbit:
\begin{itemize}
    \item \texttt{orbit\_zero}: The 0-th iterate is the starting point $z$.
    \item \texttt{orbit\_succ}: The $(n+1)$-th iterate is $f_c$ applied to the $n$-th iterate. The proof uses properties of function iteration.
\end{itemize}

\subsection{Julia and Mandelbrot Sets}

\subsubsection{Boundedness and Sets}

\begin{lstlisting}[language=Lean]
/-- Boundedness of an orbit. -/
def boundedOrbit (c z : $\mathbb{C}$) : Prop :=
  $\exists$ M : $\mathbb{R}$, $\forall$ n, $\|$orbit c z n$\|$ $\le$ M

/-- Filled Julia set. -/
def K (c : $\mathbb{C}$) : Set $\mathbb{C}$ := { z | boundedOrbit c z }

/-- Julia set as the topological boundary of `K(c)`. -/
def J (c : $\mathbb{C}$) : Set $\mathbb{C}$ := frontier (K c)

/-- The Mandelbrot set is the set of parameters `c` for which the orbit of `0` is bounded. -/
def MandelbrotSet : Set $\mathbb{C}$ := { c | boundedOrbit c 0 }
\end{lstlisting}

\textbf{Explanation:}
\begin{itemize}
    \item \texttt{boundedOrbit}: A proposition stating that the sequence of norms $|f_c^n(z)|$ is bounded by some real number $M$.
    \item \texttt{K}: The set of points $z$ with bounded orbits (filled Julia set).
    \item \texttt{J}: The topological boundary of $K$.
    \item \texttt{MandelbrotSet}: The set of parameters $c$ where the critical point $0$ has a bounded orbit ($0 \in K_c$).
\end{itemize}

\subsection{Elementary Norm Facts and Escape Radius}

\subsubsection{Norm Squared}

\begin{lstlisting}[language=Lean]
@[simp] lemma norm_sq (z : $\mathbb{C}$) : $\|$z^2$\|$ = $\|$z$\|^2$ := by
  simp [pow_two]
\end{lstlisting}

\textbf{Explanation:}
A simple helper lemma stating $|z^2| = |z|^2$.

\subsubsection{Escape Radius R(c)}

\begin{lstlisting}[language=Lean]
/-- A convenient escape radius depending on `$\|$c$\|$`. -/
def R (c : $\mathbb{C}$) : $\mathbb{R}$ := max 2 (1 + $\|$c$\|$)

@[simp] lemma R_ge_two (c : $\mathbb{C}$) : R c $\ge$ 2 := by simp [R]
@[simp] lemma R_ge_one_plus_c (c : $\mathbb{C}$) : R c $\ge$ 1 + $\|$c$\|$ := by simp [R]
\end{lstlisting}

\textbf{Explanation:}
We define the escape radius $R(c) = \max(2, 1+|c|)$. This specific choice is standard in the literature (e.g., Milnor, Lyubich) because:
\begin{itemize}
    \item $R(c) \ge 2$: Ensures $|z| > 2 \implies |z|^2 > 2|z|$, helping expansion.
    \item $R(c) > |c|$: Ensures the quadratic term dominates the constant term $c$.
\end{itemize}

\subsection{Connectivity Axioms}

\begin{lstlisting}[language=Lean]
/-- The Mandelbrot set is connected. -/
axiom mandelbrot_set_connected : IsConnected MandelbrotSet

/-- The filled Julia set is connected if c is in the Mandelbrot set. -/
axiom filled_julia_set_connected {c : $\mathbb{C}$} (h : c $\in$ MandelbrotSet) : IsConnected (K c)
\end{lstlisting}

\textbf{Explanation:}
We state two deep results from Douady and Hubbard as axioms for this formalization:
\begin{itemize}
    \item The Mandelbrot set is connected.
    \item For $c \in \mathcal{M}$, the filled Julia set $K_c$ is connected.
\end{itemize}
