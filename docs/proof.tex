\documentclass[reqno]{amsart}
\usepackage{amssymb, amsmath}
\usepackage[left=2.5cm, right=2.5cm, top=2.5cm, bottom=2.5cm]{geometry}
\usepackage{hyperref}
\usepackage{listings}
\usepackage{xcolor}
\usepackage{fontspec}
\usepackage{tikz}
\usetikzlibrary{shapes,arrows,positioning,calc}
\setmonofont{DejaVu Sans Mono}[Scale=MatchLowercase]

% Define colors for code listing
\definecolor{codegreen}{rgb}{0,0.6,0}
\definecolor{codegray}{rgb}{0.5,0.5,0.5}
\definecolor{codepurple}{rgb}{0.58,0,0.82}
\definecolor{backcolour}{rgb}{0.95,0.95,0.92}

% Listings configuration
\lstdefinelanguage{Lean}{
  keywords={def, theorem, lemma, axiom, structure, inductive, where, deriving, noncomputable, opaque, section, namespace, open, end, variable, variables, Prop, Type, Sort, if, then, else, let, in, do, case, match, with, by, intro, exact, apply, rw, simp, rcases, obtain, use, have, calc, contradiction, exfalso, constructor, split_ifs},
  keywordstyle=\color{blue}\bfseries,
  ndkeywords={class, export, boolean, throw, implements, import, this},
  ndkeywordstyle=\color{darkgray}\bfseries,
  identifierstyle=\color{black},
  sensitive=true,
  comment=[l]{--},
  morecomment=[s]{/-}{-/},
  commentstyle=\color{codegreen}\ttfamily,
  stringstyle=\color{red}\ttfamily,
  morestring=[b]"
}

\lstset{
  backgroundcolor=\color{backcolour},
  basicstyle=\ttfamily\footnotesize,
  breakatwhitespace=false,
  breaklines=true,
  captionpos=b,
  keepspaces=true,
  numbers=left,
  numbersep=5pt,
  showspaces=false,
  showstringspaces=false,
  showtabs=false,
  tabsize=2,
  extendedchars=true,
  mathescape=true,
  literate={ℂ}{$\mathbb{C}$}1
           {ℝ}{$\mathbb{R}$}1
           {ℕ}{$\mathbb{N}$}1
           {ℤ}{$\mathbb{Z}$}1
           {∀}{$\forall$}1
           {∃}{$\exists$}1
           {→}{$\to$}1
           {↔}{$\leftrightarrow$}1
           {≤}{$\le$}1
           {≥}{$\ge$}1
           {⊆}{$\subseteq$}1
           {⊂}{$\subset$}1
           {∪}{$\cup$}1
           {∩}{$\cap$}1
           {⋃}{$\bigcup$}1
           {⋂}{$\bigcap$}1
           {∈}{$\in$}1
           {∉}{$\notin$}1
           {∅}{$\emptyset$}1
           {⟨}{$\langle$}1
           {⟩}{$\rangle$}1
}

\title{Formalization of Yoccoz's Theorem for Quadratic Polynomials}
\author{\url{https://github.com/kirill-kondrashov/mlc}}
\date{\today}

\begin{document}

\begin{abstract}
  This document presents a formalization of Yoccoz's Theorem for quadratic polynomials, which is a key step towards the Local Connectivity of the Mandelbrot Set (MLC) conjecture. The formalization is implemented in Lean 4. The Lean 4 code was generated by Gemini 3 Pro Preview with manual fixes and adjustments. It is currently under manual verification; see \url{https://github.com/kirill-kondrashov/mlc/blob/main/README.md} for the latest updates.
\end{abstract}

\maketitle

\section{Introduction}

The Mandelbrot set $\mathcal{M}$ is the set of parameters $c \in \mathbb{C}$ for which the orbit of 0 under $f_c(z) = z^2 + c$ is bounded. Yoccoz's Theorem asserts that for non-renormalizable parameters, the dynamical puzzle pieces shrink to a point, which implies local connectivity at those parameters.

This formalization covers the proof of Yoccoz's Theorem ("Shrinking Puzzle Pieces") for non-renormalizable parameters.

\section{Context}

The MLC conjecture has stood as one of the most significant open problems in complex dynamics. Yoccoz established the result for non-renormalizable parameters. This work formalizes that result.

\subsection{Disclaimer and Verification}
While this proof has been checked by the Lean 4 kernel, which provides a high degree of assurance regarding the logical consistency of the derivation from the specified axioms, it is essential to emphasize that \textbf{human expert verification is still required}.

The correctness of the result depends entirely on:
\begin{itemize}
    \item The faithfulness of the formal definitions (e.g., \texttt{MandelbrotSet}, \texttt{Modulus}, \texttt{PuzzlePiece}) to their standard mathematical counterparts.
    \item The validity of the axioms introduced (e.g., properties of Green's functions, Grötzsch's inequality).
\end{itemize}


\vspace{1em}
\noindent
\fcolorbox{black}{yellow!20}{%
    \begin{minipage}{\dimexpr\linewidth-2\fboxsep-2\fboxrule}
        \centering
        \textbf{Disclaimer}: All content marked as \textbf{Proof} is just a description of a strict proof in Lean which is shown as a listing. It needs multiple readouts as a human and is used primarily to have a gist of the Lean proofs. For some most obvious cases \textbf{Proof} is omitted. Anyways, all Lean listings are accompanied by comments.
    \end{minipage}%
}
\vspace{1em}

\part*{Formalization Structure}

\section{Basic}
\textit{Source: \texttt{Quadratic/Complex/Basic.lean}}

\begin{quotation}
\textbf{Quadratic family basics (Lyubich I–II notation)}

We set up the quadratic family \texttt{f\_c(z) = z\textasciicircum{}2 + c}, iterates, filled Julia set \texttt{K(c)},
and Julia set \texttt{J(c) = $\partial$K(c)}. We also state (and stub) the standard escape and
compactness lemmas you’ll prove next.
\end{quotation}

\subsection*{fc}\label{fc}
The quadratic polynomial \texttt{f\_c(z) = z\textasciicircum{}2 + c}.

\textit{(See code: fc)}

\subsection*{orbit}\label{orbit}
The forward orbit of \texttt{z0} under \texttt{f\_c}.

\textit{(See code: orbit)}

\subsection*{boundedOrbit}\label{boundedOrbit}
Boundedness of an orbit.

\textit{(See code: boundedOrbit)}

\subsection*{K}\label{K}
Filled Julia set.

\textit{(See code: K)}

\subsection*{J}\label{J}
Julia set as the topological boundary of \texttt{K(c)}.

\textit{(See code: J)}

\subsection*{MandelbrotSet}\label{MandelbrotSet}
The Mandelbrot set is the set of parameters \texttt{c} for which the orbit of \texttt{0} is bounded.

\textit{(See code: MandelbrotSet)}

\subsection*{R}\label{R}
A convenient escape radius depending on \texttt{‖c‖}.

\textit{(See code: R)}


\section{The Escape Lemma}
\label{sec:Escape}

This section establishes the fundamental escape criterion for the quadratic family $f_c(z) = z^2 + c$. The result ensures that if the orbit of a point leaves a sufficiently large disk, it necessarily escapes to infinity.

\subsection*{Escape Radius}
Recall the definition of the escape radius from the Basic properties:
\[ R(c) = \max(2, 1 + |c|). \]
This radius provides a threshold: any point outside the disk $D(0, R(c))$ will have an orbit that grows without bound.

\begin{figure}[ht]
    \centering
    \includegraphics[width=0.6\textwidth]{images/escape_radius.png}
    \caption{The Escape Radius $R(c)$. Orbits starting outside the red circle (blue trajectory) escape to infinity. Orbits inside (green) may remain bounded.}
    \label{fig:escape_radius}
\end{figure}

\subsection*{The Escape Lemma}
\textbf{Lemma} (\texttt{escape\_lemma}).
\textit{Let $c, z \in \mathbb{C}$. If there exists an $n \in \mathbb{N}$ such that $|f_c^n(z)| > R(c)$, then the orbit of $z$ escapes to infinity:
\[ \lim_{m \to \infty} |f_c^m(z)| = \infty. \]
}

\begin{proof}[Proof Idea]
(See Lean: \texttt{MLC.Quadratic.escape\_lemma})

Let $w = f_c^n(z)$. Since $|w| > R(c)$, we define a ``growth factor'' $\lambda(w)$:
\[ \lambda(w) = |w| - \frac{|c|}{|w|}. \]
First, we show that $|w| > R(c)$ implies $\lambda(w) > 1$ (Lean: \texttt{factor\_gt\_one}).
Then, using induction on $k$, we establish the inequality:
\[ |f_c^k(w)| \ge |w| \cdot \lambda(w)^k. \]
Since $\lambda(w) > 1$, the term $\lambda(w)^k$ tends to infinity as $k \to \infty$, forcing $|f_c^k(w)| \to \infty$.
\end{proof}

\subsection*{Orbit Growth}
A direct consequence of the escape mechanism is that the norm of the orbit cannot decrease once it exceeds the escape radius.

\textbf{Lemma} (\texttt{norm\_orbit\_ge\_of\_norm\_ge\_R}).
\textit{If $|z| > R(c)$, then for all $n \in \mathbb{N}$,
\[ |f_c^n(z)| \ge |z|. \]
}

\subsection*{Significance for MLC}
The escape lemma is foundational for the Yoccoz puzzle construction:
\begin{itemize}
    \item \textbf{Basin of Infinity}: It characterizes the basin of attraction of infinity, $A(\infty)$, showing it contains the exterior of the disk of radius $R(c)$.
    \item \textbf{Green's Function}: It ensures the well-definedness of the Green's function $G_c(z) = \lim_{n \to \infty} 2^{-n} \log^+ |f_c^n(z)|$ on the basin of infinity, which in turn defines the equipotentials and external rays used to cut the puzzle pieces.
\end{itemize}

\section{Green's Function}
\label{sec:Green}

The Green's function $G_c(z)$ is a key tool in complex dynamics, providing a potential-theoretic measure of the escape rate to infinity for the quadratic map $f_c(z) = z^2 + c$.

\subsection*{Definition}

For any parameter $c \in \mathbb{C}$, the Green's function is defined as the limit of the normalized logarithmic potential:
\[
G_c(z) = \lim_{n \to \infty} \frac{1}{2^n} \log^+ |f_c^n(z)|
\]
where $\log^+(x) = \max(0, \log x)$. In Lean, this is formalized using the sequence \texttt{potential\_seq}:
\[ \text{\texttt{potential\_seq}}(c, z, n) = 2^{-n} \log |f_c^n(z)| \]
(Note: For large $z$, $|f_c^n(z)| \ge 1$, so the $\log$ is non-negative).

\begin{figure}[ht]
    \centering
    \includegraphics[width=1.0\textwidth]{images/green_function_plots.png}
    \caption{Visualization of the Green's function for $c=0$ (unit disk, left) and $c=-1$ (Basilica, right). The black region represents the filled Julia set $K(c)$ where $G_c(z)=0$. The equipotential lines in the basin of infinity are shown in white.}
    \label{fig:green_function}
\end{figure}

\subsection*{Functional Equation}
\textbf{Lemma} (\texttt{green\_function\_functional\_eq}).
\textit{The Green's function satisfies the functional equation:}
\[ G_c(f_c(z)) = 2 G_c(z). \]

\begin{figure}[ht]
    \centering
    \includegraphics[width=0.6\textwidth]{images/green_functional_eq.png}
    \caption{Functional Equation $G(f(z)) = 2G(z)$. The map $f_c$ transforms the equipotential curve at level $L=0.5$ (blue) to the curve at level $2L=1.0$ (red).}
    \label{fig:green_functional_eq}
\end{figure}

\begin{proof}[Proof Idea]
The factor of 2 comes from the squaring nature of the map near infinity:
\[
G_c(f_c(z)) = \lim_{n \to \infty} \frac{1}{2^n} \log |f_c^{n+1}(z)| = 2 \lim_{n \to \infty} \frac{1}{2^{n+1}} \log |f_c^{n+1}(z)| = 2 G_c(z).
\]
\end{proof}

\subsection*{Characterization of the Filled Julia Set}
The zero set of the Green's function precisely recovers the filled Julia set $K(c)$.

\textbf{Theorem} (\texttt{green\_function\_eq\_zero\_iff\_mem\_K}).
\textit{For any $z \in \mathbb{C}$,}
\[ G_c(z) = 0 \iff z \in K(c). \]

\begin{proof}[Proof Idea]
(See Lean: \texttt{green\_function\_eq\_zero\_iff\_mem\_K})
\begin{itemize}
    \item ($\Leftarrow$) If $z \in K(c)$, the orbit $f_c^n(z)$ is bounded. Thus $\frac{1}{2^n} \log |f_c^n(z)| \to 0$.
    \item ($\Rightarrow$) If $z \notin K(c)$, the orbit escapes to infinity. Using the escape estimates (see Section \ref{sec:Escape}), one can show the potential converges to a strictly positive value.
\end{itemize}
\end{proof}

\subsection*{Basin of Infinity}
It follows that the Green's function is strictly positive on the basin of infinity $A(\infty) = \mathbb{C} \setminus K(c)$.

\textbf{Lemma} (\texttt{green\_function\_pos\_iff\_not\_mem\_K}).
\[ G_c(z) > 0 \iff z \notin K(c). \]

\subsection*{Harmonicity}
Although not detailed in the sketched results here, the Green's function is harmonic on the domain $\mathbb{C} \setminus K(c)$. This property allows the definition of harmonic conjugate lines (equipotentials) and orthogonal gradient lines (external rays), which form the coordinate grid for Yoccoz puzzles.


\section{Properties of the Green's Function}
\label{sec:GreenLemmas}

This section details the convergence properties of the sequence defining the Green's function.

\subsection*{Potential Sequence}
The Green's function is constructed as the limit of the sequence:
\[ G_n(z) = \frac{1}{2^n} \log \max(1, |f_c^n(z)|). \]
This sequence captures the exponential growth rate of the orbit norm.

\subsection*{Convergence}
\textbf{Theorem} (\texttt{potential\_seq\_converges}).
\textit{For any $c, z \in \mathbb{C}$, the sequence $G_n(z)$ converges to a finite limit $G_c(z)$.}

\begin{proof}[Proof Idea]
The proof splits into two cases:
\begin{itemize}
    \item \textbf{Bounded Orbits ($z \in K(c)$)}: The orbit $|f_c^n(z)|$ remains bounded by some $M$. Thus,
    \[ 0 \le G_n(z) \le \frac{1}{2^n} \log(\max(1, M)) \to 0. \]
    So $G_c(z) = 0$.
    \item \textbf{Escaping Orbits ($z \notin K(c)$)}: Once the orbit exceeds the escape bound $R(c)$, the growth is approximately squaring ($|z_{n+1}| \approx |z_n|^2$). We show the sequence is Cauchy by bounding the difference:
    \[ |G_{n+1}(z) - G_n(z)| \approx \frac{1}{2^{n+1}} \left| \log \frac{|f_c(z_n)|}{|z_n|^2} \right| \le \frac{C}{2^n |z_n|^2}. \]
    Since $|z_n|$ grows exponentially, the sum of differences converges rapidly.
\end{itemize}

\begin{figure}[ht]
    \centering
    \includegraphics[width=1.0\textwidth]{images/convergence_plots.png}
    \caption{Visualization of the convergence of the potential sequence $G_n(z)$. Left: The sequence values stabilize rapidly for escaping points (orange, blue) and stay zero for bounded points (green). Right: The difference $|G_{n+1}-G_n|$ decays doubly exponentially for escaping points, illustrating the Cauchy property.}
    \label{fig:convergence}
\end{figure}

\end{proof}

\subsection*{Limit Definition}
\textbf{Lemma} (\texttt{green\_function\_eq\_lim}).
\textit{The function \texttt{green\_function} coincides with the limit of the potential sequence:}
\[ G_c(z) = \lim_{n \to \infty} G_n(z). \]
For a visualization of the limit function $G_c(z)$ and its geometry, see Figure \ref{fig:green_function}.

\subsection*{Escape Bound}
The convergence proof relies on an explicit escape bound slightly larger than $R(c)$ to ensure stable logarithmic behavior.
\textbf{Definition} (\texttt{escape\_bound}).
\[ B(c) = \max\left(R(c), \sqrt{2|c| + 1}\right). \]
This bound ensures that once $|z| > B(c)$, the term $|c|/|z|^2$ is small enough (specifically $\le 1/2$) to use Taylor expansion estimates for the logarithm.

\section{Grötzsch's Inequality and Criterion}
\label{sec:Groetzsch}

This section details the analytical core of the proof: Grötzsch's inequality. This classical result from complex analysis connects the geometry of nested annuli to the convergence properties of their moduli, providing a rigorous criterion for a sequence of sets to shrink to a single point.

\subsection*{Conformal Modulus}

The conformal modulus is a fundamental invariant that measures the "thickness" of an annulus. Geometrically, it generalizes the aspect ratio of a rectangle.

\textbf{Definition} (\texttt{modulus}).
In this formalization, we define the modulus of a set $A$ as its weighted area using a Gaussian weight function:
\[ \operatorname{mod}(A) = \int_A e^{-|z|^2} \, dz. \]
This serves as a simplified proxy for the classical conformal modulus. While it differs from the standard definition (which is conformally invariant), it satisfies the crucial superadditivity property (Grötzsch's inequality) required for the proof.
\begin{itemize}
    \item For a standard annulus $A(r, R) = \{z : r < |z| < R\}$, the classical modulus is $\frac{1}{2\pi} \log(R/r)$.
    \item The weighted area definition ensures integrability on the whole plane.
    \item For degenerate sets (like the empty set), $\operatorname{mod}(\emptyset) = 0$.
\end{itemize}

\subsection*{Grötzsch's Inequality (Proxy Form)}

The power of this modulus definition comes from the monotonicity of the integral.

\begin{lemma}[\texttt{groetzsch\_inequality}]
Let $S \subset \mathbb{C}$ be a domain. If $A, B \subset S$ are disjoint sets, then by the additivity and monotonicity of the integral:
\[ \operatorname{mod}(A) + \operatorname{mod}(B) = \int_{A \cup B} \text{weight} \le \int_S \text{weight} = \operatorname{mod}(S). \]
\end{lemma}

\begin{figure}[ht]
    \centering
    \includegraphics[width=0.6\textwidth]{images/groetzsch_packing.png}
    \caption{Grötzsch Inequality Packing. Disjoint annuli $A_1, A_2, A_3$ packed inside a larger container $S$. The sum of their "thicknesses" (moduli) cannot exceed the thickness of $S$.}
    \label{fig:groetzsch_packing}
\end{figure}

\begin{proof}[Geometric Intuition]
This inequality expresses that we cannot pack arbitrary "conformal thickness" into a finite container. If we stack disjoint annuli inside a larger annulus $S$, their total thickness (modulus) cannot exceed the thickness of $S$.
\end{proof}

\subsection*{The Summability Theorem}

We apply this inequality to a sequence of nested connected sets $P_0 \supset P_1 \supset P_2 \supset \dots$ containing $0$. These sets form the "puzzle pieces" of our construction. We are interested in the annuli formed by their differences: $A_n = P_n \setminus P_{n+1}$.

\begin{theorem}[\texttt{modulus\_summable\_of\_nontrivial\_intersection}]
If the intersection $K = \bigcap_{n} P_n$ is non-trivial (contains more than just $\{0\}$), then the sum of the moduli converges:
\[ \sum_{n=0}^{\infty} \operatorname{mod}(A_n) < \infty. \]
\end{theorem}

\begin{proof}[Proof Idea]
(See Lean: \texttt{modulus\_summable\_of\_nontrivial\_intersection})

If $K$ is non-trivial, it serves as a "hard core" inside all $P_n$. The annuli $A_n$ are thus all contained in the fixed domain $P_0 \setminus K$.
Since the $A_n$ are disjoint, Grötzsch's inequality implies that for any $N$:
\[ \sum_{n=0}^N \operatorname{mod}(A_n) \le \operatorname{mod}(P_0 \setminus K). \]
Since $K$ contains points other than $0$, the container $P_0 \setminus K$ is not the punctured plane (which would have infinite modulus) but a domain with finite modulus. Therefore, the partial sums are bounded, and the series converges.
\end{proof}

\begin{figure}[ht]
    \centering
    \includegraphics[width=0.9\textwidth]{images/groetzsch_viz.png}
    \caption{Geometric Interpretation. Left: A non-trivial core $K$ (red) limits the available space, forcing the moduli of the nested annuli $A_n$ to decay rapidly ($\sum < \infty$). Right: If the intersection is a single point $\{0\}$, there is no inner barrier, allowing the moduli sum to diverge ($\sum = \infty$).}
    \label{fig:groetzsch}
\end{figure}

\subsection*{Grötzsch's Criterion}

By taking the contrapositive of the summability theorem, we obtain a sufficient condition for the intersection to be a single point. This is the tool we use to prove Yoccoz's theorem.

\begin{theorem}[\texttt{groetzsch\_criterion}]
\label{thm:groetzsch_criterion}
Let $\{P_n\}$ be a nested sequence of connected sets containing $0$. If the sum of the moduli of the difference annuli diverges:
\[ \sum_{n=0}^{\infty} \operatorname{mod}(P_n \setminus P_{n+1}) = \infty, \]
then the intersection consists only of the point $0$:
\[ \bigcap_{n=0}^{\infty} P_n = \{0\}. \]
\end{theorem}

\begin{proof}[Proof Idea]
(See Lean: \texttt{groetzsch\_criterion})

Assume for contradiction that the intersection is not $\{0\}$. Since $0$ is in every set, the intersection must contain some other point $z \ne 0$. This implies the intersection is non-trivial.
By the Summability Theorem above, a non-trivial intersection implies that $\sum \operatorname{mod}(A_n) < \infty$.
But this contradicts the hypothesis that the sum is infinite.
Therefore, the intersection must be exactly $\{0\}$.
\end{proof}

\subsection*{Significance for MLC}
This criterion bridges the gap between combinatorics and topology:
\begin{itemize}
    \item \textbf{Combinatorics}: We will show that for non-renormalizable parameters, the puzzle pieces recur frequently enough to generate infinite total modulus.
    \item \textbf{Topology}: Grötzsch's Criterion then guarantees that the pieces shrink to a point, proving local connectivity at the critical parameter $c$.
\end{itemize}


\section{Puzzle}
\textit{Source: \texttt{Quadratic/Complex/Puzzle.lean}}

This module contains the formal definitions and proofs.
\subsection*{DynamicalPuzzlePiece}\label{DynamicalPuzzlePiece}
The dynamical puzzle piece of depth n containing z.
    Definition: The connected component of \texttt{\{z | G(z) < 1/2\textasciicircum{}n\}} containing \texttt{z}.
    These pieces form the basis of the Yoccoz puzzle construction.

\textit{(See code: DynamicalPuzzlePiece)}

\subsection*{PuzzleAnnulus}\label{PuzzleAnnulus}
The annulus between two nested puzzle pieces around the critical point.
    Definition: The annulus between two nested puzzle pieces \texttt{P\_n \textbackslash{} P\_\{n+1\}}.
    The sum of their moduli determines renormalizability.

\textit{(See code: PuzzleAnnulus)}

\subsection*{ParaPuzzlePiece}\label{ParaPuzzlePiece}
A para-puzzle piece in the parameter plane.
    Definition: The set of parameters \texttt{c} for which the critical point 0 lies in the
    dynamical puzzle piece of depth \texttt{n}.

\textit{(See code: ParaPuzzlePiece)}


\section{Puzzle Properties}
\label{sec:PuzzleLemmas}

This section establishes the basic topological properties of the puzzle pieces defined in \ref{sec:Puzzle}.

\subsection{Nesting Property}

A crucial feature of the puzzle construction is that pieces at higher depth are contained in pieces of lower depth.

\begin{lemma}[Nesting]
For any $n \in \mathbb{N}$, the puzzle piece at depth $n+1$ is contained in the piece at depth $n$:
\[ P_{n+1}(0) \subseteq P_n(0). \]
\end{lemma}

\begin{proof}[Proof Idea]
(See Lean: \texttt{dynamical\_puzzle\_piece\_nested})

The puzzle piece $P_n(0)$ is a component of the sublevel set $U_n = \{ w \mid G(w) < 2^{-n} \}$.
Since $2^{-(n+1)} < 2^{-n}$, the sublevel set $U_{n+1}$ is a subset of $U_n$.
The connected component of a point in a subset is necessarily contained in the connected component of that point in the superset.
\end{proof}

\subsection{Behavior for Mandelbrot Set}

For parameters in the Mandelbrot set, the critical point never escapes, so it remains deep inside the puzzle pieces.

\begin{lemma}[Persistence]
If $c \in \mathcal{M}$, then for all $n$, the critical point $0$ is contained in $P_n(0)$.
\end{lemma}

\begin{proof}[Proof Idea]
(See Lean: \texttt{mem\_dynamical\_puzzle\_piece\_self})

If $c \in \mathcal{M}$, then the orbit of 0 is bounded, so $0$ lies in the filled Julia set $K(c)$.
By the property of the Green's function, $z \in K(c) \iff G_c(z) = 0$.
Since $0 < 2^{-n}$ for all $n$, $0$ satisfies the condition $G_c(0) < 2^{-n}$ and is thus in the puzzle piece.
\end{proof}

\subsection{Behavior for Escaping Parameters}

Conversely, if the parameter is outside the Mandelbrot set, the Green's function at the critical point is positive, and eventually the puzzle pieces exclude it.

\begin{lemma}[Eventual Empty]
If $c \notin \mathcal{M}$, there exists some $N$ such that for all $n \ge N$,
\[ 0 \notin P_n(0). \]
\end{lemma}

\begin{proof}[Proof Idea]
(See Lean: \texttt{dynamical\_puzzle\_piece\_empty\_of\_large\_n})

If $c \notin \mathcal{M}$, then $G_c(0) > 0$.
The sequence $2^{-n}$ converges to 0.
Therefore, for sufficiently large $n$, we have $2^{-n} < G_c(0)$.
The condition $G_c(0) < 2^{-n}$ fails, so $0$ is not in the sublevel set, and thus not in $P_n(0)$.
\end{proof}

\section{Yoccoz's Theorem}
\label{sec:Yoccoz}

This section describes the verification of the file \texttt{Mlc/Yoccoz.lean}. It proves the central result for non-renormalizable parameters: the divergence of puzzle moduli implies the shrinking of puzzle pieces.

\subsection{Non-Renormalizability}

We define non-renormalizability via the combinatorial condition of divergent moduli.

\begin{lstlisting}[language=Lean]
/-- Non-renormalizable parameters.
    For the purpose of this plan, we define non-renormalizable parameters
    as those for which the Yoccoz puzzle moduli diverge.
    The deep work is then in the dichotomy axiom. -/
def NonRenormalizable (c : ℂ) : Prop :=
    ¬ Summable (fun n => modulus (PuzzleAnnulus c n))

/-- Non-renormalizable parameters have divergent moduli. -/
theorem non_renormalizable_moduli_diverge (c : ℂ) (h : NonRenormalizable c) :
    ¬ (Summable fun n => modulus (PuzzleAnnulus c n)) := h
\end{lstlisting}

\noindent
\textbf{Simple proof idea:}
\begin{itemize}
    \item The definition \texttt{NonRenormalizable c} is exactly the proposition that the sequence of annulus moduli is not summable.
    \item The Lean theorem \texttt{non\_renormalizable\_moduli\_diverge} is therefore a one-line wrapper that returns the hypothesis: given \texttt{h : NonRenormalizable c}, the conclusion follows immediately by returning \texttt{h}.
\end{itemize}

\begin{proof}
Immediate from the definition: the hypothesis \texttt{h} is the required conclusion, so the proof is simply \texttt{exact h}.
\end{proof}

\noindent
\textbf{Detailed Explanation:}
\begin{itemize}
    \item \texttt{NonRenormalizable c}: Defined as the negation of summability for the moduli of puzzle annuli.
    \item This definition serves as the hypothesis for applying Grötzsch's criterion.
\end{itemize}

\subsection{Yoccoz's Theorem}

The main theorem establishes that for these parameters, the dynamical puzzle pieces around the critical point shrink to exactly $\{0\}$.

\begin{lstlisting}[language=Lean]
theorem yoccoz_theorem (c : ℂ) :
    ¬ (Summable fun n => modulus (PuzzleAnnulus c n)) →
    (⋂ n, DynamicalPuzzlePiece c n 0) = {0} := by
  intro h_div
  by_cases hc : c ∈ MandelbrotSet
  · apply groetzsch_criterion
    · intro n
      apply dynamical_puzzle_piece_nested
    · intro n
      apply mem_dynamical_puzzle_piece_self c hc
    · intro n
      have h_ne : (DynamicalPuzzlePiece c n 0).Nonempty := ⟨0, mem_dynamical_puzzle_piece_self c hc n⟩
      rw [DynamicalPuzzlePiece] at h_ne ⊢
      exact ⟨h_ne, isPreconnected_connectedComponentIn⟩
    · exact h_div
  · exfalso
    apply h_div
    rcases dynamical_puzzle_piece_empty_of_large_n c hc with ⟨N, hN⟩
    apply summable_of_finite_support
    have : (Function.support fun n ↦ modulus (PuzzleAnnulus c n)) ⊆ Iio N := by
      intro n hn
      rw [Function.mem_support, ne_eq] at hn
      by_contra h_ge
      simp at h_ge
      have : modulus (PuzzleAnnulus c n) = 0 := by
        rw [PuzzleAnnulus]
        have h_empty : DynamicalPuzzlePiece c n 0 = ∅ := by
          ext x
          simp
          intro hx
          have h0 : 0 ∈ DynamicalPuzzlePiece c n 0 := by
            rw [DynamicalPuzzlePiece] at hx ⊢
            apply mem_connectedComponentIn
            exact connectedComponentIn_nonempty_iff.1 ⟨x, hx⟩
          exact hN n h_ge h0
        rw [h_empty]
        simp
        exact modulus_empty
      contradiction
    exact Set.Finite.subset (Set.finite_Iio N) this
\end{lstlisting}

\noindent
\textbf{Detailed Explanation:}
\begin{itemize}
    \item \textbf{Goal}: Prove $\bigcap P_n(0) = \{0\}$ assuming divergence of moduli.
    \item \textbf{Case 1: $c \in \mathcal{M}$}:
        \begin{itemize}
            \item We apply \texttt{groetzsch\_criterion}.
            \item \textbf{Hypotheses Check}:
                \begin{itemize}
                    \item \texttt{nested}: Verified by \texttt{dynamical\_puzzle\_piece\_nested}.
                    \item \texttt{contains 0}: Verified by \texttt{mem\_dynamical\_puzzle\_piece\_self}.
                    \item \texttt{connected}: Verified by construction (connected components).
                    \item \texttt{divergence}: Assumed by hypothesis \texttt{h\_div}.
                \end{itemize}
            \item The criterion then immediately yields the result.
        \end{itemize}
    \item \textbf{Case 2: $c \notin \mathcal{M}$}:
        \begin{itemize}
            \item We derive a contradiction.
            \item If $c \notin \mathcal{M}$, then the puzzle pieces eventually become empty (\texttt{dynamical\_\allowbreak puzzle\_\allowbreak piece\_\allowbreak empty\_\allowbreak of\_\allowbreak large\_n}).
            \item This means the annuli become empty for large $n$, so their moduli are 0.
            \item A sequence of non-negative terms that is eventually 0 has a finite sum (\texttt{summable\_\allowbreak of\_\allowbreak finite\_\allowbreak support}).
            \item This contradicts the divergence hypothesis \texttt{h\_div}.
            \item Thus, non-renormalizable parameters (by this definition) must lie in the Mandelbrot set.
        \end{itemize}
\end{itemize}

\begin{proof}
Simplified proof: If $c\in\mathcal{M}$, apply Grötzsch's criterion to the nested, connected dynamical pieces that all contain $0$; divergence of the annulus moduli forces their intersection to be the single point $\{0\}$. If $c\notin\mathcal{M}$ the pieces are eventually empty, so the moduli vanish and their sum is finite, contradicting the divergence hypothesis. Therefore only the first case can occur and the intersection equals $\{0\}$.
\end{proof}


\section{Logical Dependencies}

This section describes the logical flow of the proof.

\subsection{The Foundation}
The proof rests on the basic dynamical definitions provided in \texttt{Basic.tex}. The geometric framework is built upon the Green's function (\texttt{Green.tex}, \texttt{GreenLemmas.tex}), which provides a coordinate system for the escaping set. This allows the construction of \textbf{Puzzle Pieces} (\texttt{Puzzle.tex}), which are the fundamental building blocks of the proof.

\subsection{The Analytic Core: Grötzsch's Inequality}
The engine of the proof is \textbf{Grötzsch's Inequality} (\texttt{Groetzsch.tex}). This classical result from complex analysis relates the geometry of nested annuli to the divergence of their moduli.
\begin{itemize}
    \item \textbf{Criterion}: The theorem \texttt{groetzsch\_criterion} establishes that if the sum of moduli of a sequence of nested annuli diverges, then the intersection of the sets they enclose must be a single point.
    \item \textbf{Role}: This transforms the topological problem ("do puzzle pieces shrink to a point?") into an analytical/combinatorial problem ("do the annuli have enough modulus?").
\end{itemize}

\subsection{The Bridge: Yoccoz's Theorem}
\texttt{Yoccoz.tex} connects the puzzle geometry with the Grötzsch criterion.
\begin{itemize}
    \item The theorem \texttt{yoccoz\_theorem} asserts that for non-renormalizable parameters, the combinatorial structure of the puzzle pieces ensures that the annuli formed by differences of pieces have sufficient modulus.
    \item Consequently, the sum of moduli diverges, and by \texttt{groetzsch\_criterion}, the dynamical puzzle pieces shrink to the critical point: $\bigcap D_n(0) = \{0\}$.
\end{itemize}

\begin{thebibliography}{9}

\bibitem{Milnor}
John Milnor, \textit{Dynamics in One Complex Variable}, arXiv:math/9201272.

\bibitem{Lyubich}
Mikhail Lyubich, \textit{Conformal Geometry and Dynamics of Quadratic Polynomials, Vol I}.

\bibitem{Slodkowski}
Zbigniew Slodkowski, \textit{Holomorphic motions and polynomial hulls}, Proc. Amer. Math. Soc. 111 (1991), 347-355.

\end{thebibliography}

\end{document}
