\section{Green}
\textit{Source: \texttt{Quadratic/Complex/Green.lean}}

\begin{quotation}
\textbf{Green's Function for the Quadratic Family}

This file defines the Green's function \texttt{G\_c(z)} for the filled Julia set \texttt{K(c)}.
The Green's function measures the rate of escape to infinity.

\textbf{Connection to MLC}

The Green's function is used to construct Yoccoz puzzles, which are central to the proof of the
Mandelbrot Local Connectivity (MLC) conjecture.

*   **Equipotentials and Rays**: Level sets of \texttt{G\_c} (equipotentials) and their orthogonal trajectories
    (external rays) form a grid on \texttt{$\mathbb{C}$ \textbackslash{} K(c)}.
*   **Yoccoz Puzzles**: Intersections of these curves define puzzle pieces used to analyze the combinatorics
    of orbits.
*   **Böttcher Coordinates**: \texttt{G\_c} is the real part of the Böttcher coordinate, conjugating \texttt{f\_c} to \texttt{z $\mapsto$ z\textasciicircum{}2}
    near infinity.

\textbf{Main Definitions}

* \texttt{potential\_seq c z n}: The sequence \texttt{1/2\textasciicircum{}n * log ‖f\_c\textasciicircum{}n(z)‖}.
* \texttt{green\_function c z}: The limit of \texttt{potential\_seq} as \texttt{n $\to$ ∞}.

\textbf{Main Results (Sketched)}

* \texttt{green\_function\_eq\_zero\_iff\_mem\_K}: \texttt{G\_c(z) = 0 $\leftrightarrow$ z $\in$ K(c)}.
* \texttt{green\_function\_functional\_eq}: \texttt{G\_c(f\_c(z)) = 2 * G\_c(z)}.
* \texttt{green\_function\_harmonic}: \texttt{G\_c} is harmonic on \texttt{$\mathbb{C}$ \textbackslash{} K(c)}.
\end{quotation}

\subsection*{green\_function\_functional\_eq}\label{green_function_functional_eq}
The Green's function satisfies the functional equation \texttt{G(f(z)) = 2 * G(z)}.
    Proof idea:
    \texttt{G(f(z)) = lim\_\{n$\to$∞\} 1/2\textasciicircum{}n log |f\textasciicircum{}n(f(z))|}
    \texttt{= lim\_\{n$\to$∞\} 1/2\textasciicircum{}n log |f\textasciicircum{}\{n+1\}(z)|}
    \texttt{= lim\_\{n$\to$∞\} 2 * (1/2\textasciicircum{}\{n+1\} log |f\textasciicircum{}\{n+1\}(z)|)}
    \texttt{= 2 * G(z)}.

\textit{(See code: green\_function\_functional\_eq)}

\subsection*{green\_function\_nonneg}\label{green_function_nonneg}
The Green's function is non-negative.

\textit{(See code: green\_function\_nonneg)}

\subsection*{green\_function\_eq\_zero\_iff\_mem\_K}\label{green_function_eq_zero_iff_mem_K}
A point is in the filled Julia set iff its Green's function is zero.
    Proof idea:
    *   \texttt{$\to$}: If \texttt{G(z) = 0}, suppose \texttt{z $\notin$ K(c)}. Then \texttt{z} escapes, and we can show \texttt{G(z) > 0}
        (using the positive growth rate outside \texttt{K(c)}). Contradiction.
    *   \texttt{←}: If \texttt{z $\in$ K(c)}, the orbit is bounded, so \texttt{potential\_seq} converges to 0.

\textit{(See code: green\_function\_eq\_zero\_iff\_mem\_K)}

\subsection*{green\_function\_pos\_iff\_not\_mem\_K}\label{green_function_pos_iff_not_mem_K}
The Green's function is positive on the basin of infinity.

\textit{(See code: green\_function\_pos\_iff\_not\_mem\_K)}

