\section{Green's Function}
\label{sec:Green}

The Green's function $G_c(z)$ is a key tool in complex dynamics, providing a potential-theoretic measure of the escape rate to infinity for the quadratic map $f_c(z) = z^2 + c$.

\subsection*{Definition}

For any parameter $c \in \mathbb{C}$, the Green's function is defined as the limit of the normalized logarithmic potential:
\[
G_c(z) = \lim_{n \to \infty} \frac{1}{2^n} \log^+ |f_c^n(z)|
\]
where $\log^+(x) = \max(0, \log x)$. In Lean, this is formalized using the sequence \texttt{potential\_seq}:
\[ \text{\texttt{potential\_seq}}(c, z, n) = 2^{-n} \log |f_c^n(z)| \]
(Note: For large $z$, $|f_c^n(z)| \ge 1$, so the $\log$ is non-negative).

\begin{figure}[ht]
    \centering
    \includegraphics[width=1.0\textwidth]{images/green_function_plots.png}
    \caption{Visualization of the Green's function for $c=0$ (unit disk, left) and $c=-1$ (Basilica, right). The black region represents the filled Julia set $K(c)$ where $G_c(z)=0$. The equipotential lines in the basin of infinity are shown in white.}
    \label{fig:green_function}
\end{figure}

\subsection*{Functional Equation}
\textbf{Lemma} (\texttt{green\_function\_functional\_eq}).
\textit{The Green's function satisfies the functional equation:}
\[ G_c(f_c(z)) = 2 G_c(z). \]

\begin{figure}[ht]
    \centering
    \includegraphics[width=0.6\textwidth]{images/green_functional_eq.png}
    \caption{Functional Equation $G(f(z)) = 2G(z)$. The map $f_c$ transforms the equipotential curve at level $L=0.5$ (blue) to the curve at level $2L=1.0$ (red).}
    \label{fig:green_functional_eq}
\end{figure}

\begin{proof}[Proof Idea]
The factor of 2 comes from the squaring nature of the map near infinity:
\[
G_c(f_c(z)) = \lim_{n \to \infty} \frac{1}{2^n} \log |f_c^{n+1}(z)| = 2 \lim_{n \to \infty} \frac{1}{2^{n+1}} \log |f_c^{n+1}(z)| = 2 G_c(z).
\]
\end{proof}

\subsection*{Characterization of the Filled Julia Set}
The zero set of the Green's function precisely recovers the filled Julia set $K(c)$.

\textbf{Theorem} (\texttt{green\_function\_eq\_zero\_iff\_mem\_K}).
\textit{For any $z \in \mathbb{C}$,}
\[ G_c(z) = 0 \iff z \in K(c). \]

\begin{proof}[Proof Idea]
(See Lean: \texttt{green\_function\_eq\_zero\_iff\_mem\_K})
\begin{itemize}
    \item ($\Leftarrow$) If $z \in K(c)$, the orbit $f_c^n(z)$ is bounded. Thus $\frac{1}{2^n} \log |f_c^n(z)| \to 0$.
    \item ($\Rightarrow$) If $z \notin K(c)$, the orbit escapes to infinity. Using the escape estimates (see Section \ref{sec:Escape}), one can show the potential converges to a strictly positive value.
\end{itemize}
\end{proof}

\subsection*{Basin of Infinity}
It follows that the Green's function is strictly positive on the basin of infinity $A(\infty) = \mathbb{C} \setminus K(c)$.

\textbf{Lemma} (\texttt{green\_function\_pos\_iff\_not\_mem\_K}).
\[ G_c(z) > 0 \iff z \notin K(c). \]

\subsection*{Harmonicity}
Although not detailed in the sketched results here, the Green's function is harmonic on the domain $\mathbb{C} \setminus K(c)$. This property allows the definition of harmonic conjugate lines (equipotentials) and orthogonal gradient lines (external rays), which form the coordinate grid for Yoccoz puzzles.

