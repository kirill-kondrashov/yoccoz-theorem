\section{Escape}
\textit{Source: \texttt{Quadratic/Complex/Escape.lean}}

\begin{quotation}
\textbf{Escape Lemma for the Quadratic Family}

This file proves the escape lemma: if the orbit of a point ever leaves the disk of
radius \texttt{R(c)}, then it escapes to infinity.

Reference: "Conformal Geometry and Dynamics of Quadratic Polynomials",
https://indico.ictp.it/event/a12182/session/2/contribution/1/material/0/0.pdf
(See Section 21.2, p. 125 for the escape lemma)

\textbf{Connection to MLC}

The escape lemma is the foundational result for the construction of Yoccoz puzzles,
which are central to the proof of the Mandelbrot Local Connectivity (MLC) conjecture.

1.  **Basin of Infinity**: It establishes that the basin of attraction of infinity,
    \$A(\textbackslash{}infty) = \textbackslash{}\{z \textbackslash{}mid f\_c\textasciicircum{}n(z) \textbackslash{}to \textbackslash{}infty\textbackslash{}\}\$, contains the complement of the disk \$D(0, R(c))\$.
2.  **Green's Function**: This allows defining the Green's function \$G(z) = \textbackslash{}lim \textbackslash{}frac\{1\}\{2\textasciicircum{}n\} \textbackslash{}log|f\_c\textasciicircum{}n(z)|\$
    for \$z \textbackslash{}in A(\textbackslash{}infty)\$.
3.  **Böttcher Coordinates**: The Green's function is used to construct the Böttcher coordinate
    \$\textbackslash{}phi\_c(z)\$ near infinity, which conjugates \$f\_c\$ to \$z \textbackslash{}mapsto z\textasciicircum{}2\$.
4.  **Rays and Equipotentials**: The level sets of \$G(z)\$ (equipotentials) and the gradient lines
    of \$G(z)\$ (external rays) form the grid for the Yoccoz puzzle partition.

Thus, this lemma justifies the existence of the dynamical plane structures required for the
combinatorial analysis in the MLC proof.
\end{quotation}

\subsection*{escape\_lemma}\label{escape_lemma}
Escape lemma: if the orbit of z ever leaves the disk of radius R(c), then it
escapes to infinity.

Proof idea:
We define a 'growth factor' \texttt{$\lambda$(w) = |w| - |c|/|w|}.
We show that if \texttt{|w| > R(c)}, then \texttt{$\lambda$(w) > 1}.
We then prove by induction that \texttt{|f\_c\textasciicircum{}k(w)| $\ge$ |w| * $\lambda$(w)\textasciicircum{}k}.
Since \texttt{$\lambda$(w) > 1}, the term \texttt{$\lambda$(w)\textasciicircum{}k} grows to infinity, so the orbit norm grows unbounded.

\textit{(See code: escape\_lemma)}

