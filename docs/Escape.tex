\section{Escape Lemma for the Quadratic Family}
\label{sec:Escape}

This section describes the verification of the file \texttt{Mlc/Quadratic/Complex/Escape.lean}. This file proves the escape lemma: if the orbit of a point ever leaves the disk of radius $R(c)$, then it escapes to infinity.

Reference: "Conformal Geometry and Dynamics of Quadratic Polynomials", Section 21.2, p. 125.

\subsection{Connection to MLC}

The escape lemma is the foundational result for the construction of Yoccoz puzzles, which are central to the proof of the Mandelbrot Local Connectivity (MLC) conjecture.
\begin{enumerate}
    \item \textbf{Basin of Infinity}: It establishes that the basin of attraction of infinity contains the complement of the disk $D(0, R(c))$.
    \item \textbf{Green's Function}: This allows defining the Green's function $G(z) = \lim \frac{1}{2^n} \log|f_c^n(z)|$.
    \item \textbf{Böttcher Coordinates}: Used to construct coordinates near infinity.
    \item \textbf{Rays and Equipotentials}: Form the grid for the Yoccoz puzzle partition.
\end{enumerate}

\subsection{Formalization}

We assume the standard imports and open the necessary namespaces.

\begin{lstlisting}[language=Lean]
namespace MLC.Quadratic
open scoped Complex
open Complex Topology Filter Real
noncomputable section
variable {c z : $\mathbb{C}$}
\end{lstlisting}

\subsubsection{Basic Inequality}

\begin{lstlisting}[language=Lean]
lemma norm_sq_sub_norm_c_ge (c z : $\mathbb{C}$) : $\|$z^2$\|$ - $\|$c$\|$ $\ge$ $\|$z$\|$^2 - (R c - 1) := by
  simp only [norm_sq]
  linarith [R_ge_one_plus_c c]
\end{lstlisting}

\textbf{Explanation of \texttt{norm\_sq\_sub\_norm\_c\_ge}:}
\begin{itemize}
    \item \texttt{simp only [norm\_sq]}: We simplify $|z^2|$ to $|z|^2$ using the property $|z^2| = |z|^2$.
    \item \texttt{linarith [R\_ge\_one\_plus\_c c]}: We use linear arithmetic with the hypothesis $R(c) \ge 1 + |c|$.
    \begin{itemize}
        \item From $R(c) \ge 1 + |c|$, we get $|c| \le R(c) - 1$.
        \item Therefore, $-|c| \ge -(R(c) - 1)$.
        \item Adding $|z|^2$ to both sides yields $|z|^2 - |c| \ge |z|^2 - (R(c) - 1)$.
    \end{itemize}
\end{itemize}

\begin{proof}
Since $|z^2|=|z|^2$ we have $|z^2|-|c|=|z|^2-|c|$. From $R(c)\ge 1+|c|$ we get $|c|\le R(c)-1$, hence
$|z|^2-|c|\ge |z|^2-(R(c)-1)$, which is exactly the asserted inequality.
\end{proof}

\subsubsection{Norm Growth Estimate}

\begin{lstlisting}[language=Lean]
lemma norm_growth (c z : $\mathbb{C}$) : $\|$fc c z$\|$ $\ge$ $\|$z$\|$^2 - (R c - 1) := by
  rw [fc]
  have h_tri : $\|$z^2 + c$\|$ $\ge$ $\|$z^2$\|$ - $\|$c$\|$ := by
    have := norm_add_le (z^2 + c) (-c)
    simp only [add_neg_cancel_right, norm_neg] at this
    linarith
  rw [norm_sq] at h_tri
  linarith [R_ge_one_plus_c c]
\end{lstlisting}

\textbf{Explanation of \texttt{norm\_growth}:}
\begin{itemize}
    \item \texttt{rw [fc]}: Expand the definition $f_c(z) = z^2 + c$.
    \item \texttt{have h\_tri}: We prove the reverse triangle inequality $|z^2+c| \ge |z^2| - |c|$.
        \begin{itemize}
            \item \texttt{norm\_add\_le}: Standard triangle inequality $|(z^2+c) + (-c)| \le |z^2+c| + |-c|$.
            \item \texttt{simp}: Simplifies to $|z^2| \le |z^2+c| + |c|$.
            \item \texttt{linarith}: Rearranges to $|z^2+c| \ge |z^2| - |c|$.
        \end{itemize}
    \item \texttt{rw [norm\_sq]}: Replace $|z^2|$ with $|z|^2$.
    \item \texttt{linarith}: Combine $|z^2+c| \ge |z|^2 - |c|$ with the fact $|c| \le R(c) - 1$ to get $|z^2+c| \ge |z|^2 - (R(c)-1)$.
\end{itemize}

\begin{proof}
Writing $f_c(z)=z^2+c$ and applying the triangle inequality gives $|z^2+c|\ge|z^2|-|c|=|z|^2-|c|$. Using $|c|\le R(c)-1$ (from $R(c)\ge1+|c|$) yields
$|f_c(z)|\ge|z|^2-(R(c)-1)$, as required.
\end{proof}

\subsubsection{Refined Lower Bound}

\begin{lstlisting}[language=Lean]
lemma norm_fc_ge_norm_sq_sub_norm_c (c z : $\mathbb{C}$) : $\|$fc c z$\|$ $\ge$ $\|$z$\|$^2 - $\|$c$\|$ := by
  rw [fc]
  have := norm_add_le (z^2 + c) (-c)
  simp only [add_neg_cancel_right, norm_neg] at this
  rw [norm_sq] at this
  linarith
\end{lstlisting}

\textbf{Explanation of \texttt{norm\_fc\_ge\_norm\_sq\_sub\_norm\_c}:}
This is a more direct application of the reverse triangle inequality without substituting $R(c)$.
\begin{itemize}
    \item It establishes $|f_c(z)| \ge |z|^2 - |c|$.
    \item This form is crucial for factoring out $|z|$ later: $|z|^2 - |c| = |z|(|z| - |c|/|z|)$.
\end{itemize}

\begin{proof}
Apply the triangle inequality in reverse: $|z^2+c|\ge|z^2|-|c|$. Since $|z^2|=|z|^2$ the inequality becomes $|f_c(z)|\ge|z|^2-|c|$, which is the claimed bound.
\end{proof}

\subsubsection{Monotonicity of Growth Factor}

\begin{lstlisting}[language=Lean]
lemma sub_div_mono (c : $\mathbb{C}$) : StrictMonoOn (fun x : $\mathbb{R}$ => x - $\|$c$\|$ / x) (Set.Ioi 0) := by
  intro x hx y _ hxy
  dsimp
  apply add_lt_add_of_lt_of_le hxy
  rw [neg_le_neg_iff]
  apply div_le_div_of_nonneg_left (norm_nonneg c) hx (le_of_lt hxy)
\end{lstlisting}

\textbf{Explanation of \texttt{sub\_div\_mono}:}
We prove that $g(x) = x - |c|/x$ is strictly increasing for $x > 0$.
\begin{itemize}
    \item \texttt{intro x hx y \_ hxy}: Let $0 < x < y$.
    \item \texttt{add\_lt\_add...}: We split the inequality into proving $x < y$ (which is true by assumption) and $-|c|/x \le -|c|/y$.
    \item \texttt{neg\_le\_neg\_iff}: Proving $-|c|/x \le -|c|/y$ is equivalent to $|c|/x \ge |c|/y$.
    \item \texttt{div\_le\_div...}: Since $x < y$, $1/x > 1/y$. With $|c| \ge 0$, multiplying preserves the order (or equality if $c=0$), so $|c|/x \ge |c|/y$.
\end{itemize}

\begin{proof}
For $0<x<y$ we have $|c|/x\ge|c|/y$ because $1/x>1/y$ and $|c|\ge0$. Hence
$(x-|c|/x)<(y-|c|/y)$, so $g(x)<g(y)$, showing $g$ is strictly increasing on $(0,\infty)$.
\end{proof}


\subsubsection{Factor Greater Than One}

\begin{lstlisting}[language=Lean]
lemma factor_gt_one (c z : $\mathbb{C}$) (h : $\|$z$\|$ > R c) : $\|$z$\|$ - $\|$c$\|$ / $\|$z$\|$ > 1 := by
  have R_pos : R c > 0 := by linarith [R_ge_two c]
  have z_pos : $\|$z$\|$ > 0 := by linarith
  have key : $\|$z$\|$ - $\|$c$\|$ / $\|$z$\|$ > R c - $\|$c$\|$ / R c := sub_div_mono c R_pos z_pos h
  have R_ge : R c - $\|$c$\|$ / R c $\ge$ 1 := by
    have hR2 : R c $\ge$ 2 := R_ge_two c
    have hRc : R c $\ge$ 1 + $\|$c$\|$ := R_ge_one_plus_c c
    rw [ge_iff_le, le_sub_iff_add_le, add_comm 1, $\leftarrow$ le_sub_iff_add_le]
    rw [div_le_iff₀ R_pos]
    have h_calc := calc
      $\|$c$\|$ = 1 * $\|$c$\|$ := by simp
      _ $\le$ (R c - 1) * (R c - 1) := by
        nlinarith
      _ $\le$ (R c - 1) * R c := by
        apply mul_le_mul_of_nonneg_left
        · linarith
        · linarith
      _ = R c * R c - R c := by ring
    linarith
  linarith
\end{lstlisting}

\textbf{Explanation of \texttt{factor\_gt\_one}:}
We show that if $|z| > R(c)$, then $|z| - |c|/|z| > 1$.
\begin{itemize}
    \item \texttt{sub\_div\_mono}: Since $|z| > R(c) > 0$ and the function $x - |c|/x$ is monotonic, we have $|z| - |c|/|z| > R(c) - |c|/R(c)$.
    \item \textbf{The Lower Bound}: We need to prove $R(c) - |c|/R(c) \ge 1$.
        \item Rearranging, this is equivalent to $R(c) - 1 \ge |c|/R(c)$, or $R(c)(R(c)-1) \ge |c|$.
        \item Since $R(c) \ge 1+|c|$, we have $R(c)-1 \ge |c|$.
        \item So the product $R(c)(R(c)-1) \ge R(c)|c| \ge 2|c| \ge |c|$. (The formalized proof uses a slightly different calc block essentially establishing the same inequality $R(c)^2 - R(c) \ge |c|$).
\end{itemize}

\begin{proof}
By monotonicity we have $|z|-|c|/|z|>R(c)-|c|/R(c)$. It therefore suffices to check
$R(c)-|c|/R(c)\ge1$, equivalently $R(c)(R(c)-1)\ge|c|$. Since $R(c)\ge1+|c|$ we get $R(c)-1\ge|c|$, and multiplying by $R(c)\ge1$ yields
$R(c)(R(c)-1)\ge R(c)|c|\ge|c|$, proving the claim.
\end{proof}

\subsubsection{Escape Lemma}

\begin{lstlisting}[language=Lean]
lemma escape_lemma (n : $\mathbb{N}$) (h : $\|$orbit c z n$\|$ > R c) :
    $\forall$ M : $\mathbb{R}$, $\exists$ N : $\mathbb{N}$, $\forall$ m $\ge$ N, $\|$orbit c z m$\|$ > M := by
  let w := orbit c z n
  let lam := $\|$w$\|$ - $\|$c$\|$ / $\|$w$\|$
  have hlam : lam > 1 := factor_gt_one c w h
  have hw_pos : $\|$w$\|$ > 0 := lt_trans (R_pos c) h

  have growth : $\forall$ k, $\|$orbit c w k$\|$ $\ge$ $\|$w$\|$ * lam ^ k := by
    intro k
    induction k with
    | zero => simp
    | succ k ih =>
      let z_k := orbit c w k
      have h_zk_ge : $\|$z_k$\|$ $\ge$ $\|$w$\|$ := by
        calc $\|$z_k$\|$ $\ge$ $\|$w$\|$ * lam ^ k := ih
          _ $\ge$ $\|$w$\|$ * 1 := by
            gcongr
            exact one_le_pow₀ (le_of_lt hlam)
          _ = $\|$w$\|$ := by simp

      have h_zk_pos : $\|$z_k$\|$ > 0 := lt_of_lt_of_le hw_pos h_zk_ge

      calc $\|$orbit c w (k + 1)$\|$
        _ = $\|$fc c z_k$\|$ := by rw [orbit_succ]
        _ $\ge$ $\|$z_k$\|^2$ - $\|$c$\|$ := norm_fc_ge_norm_sq_sub_norm_c c z_k
        _ = $\|$z_k$\|$ * ($\|$z_k$\|$ - $\|$c$\|$ / $\|$z_k$\|$) := by field_simp [h_zk_pos.ne']
        _ $\ge$ $\|$z_k$\|$ * lam := by
          gcongr
          apply (sub_div_mono c).monotoneOn
          · exact hw_pos
          · exact h_zk_pos
          · exact h_zk_ge
        _ $\ge$ ($\|$w$\|$ * lam ^ k) * lam := by
          apply mul_le_mul_of_nonneg_right ih
          exact le_of_lt (lt_trans zero_lt_one hlam)
        _ = $\|$w$\|$ * lam ^ (k + 1) := by
          rw [pow_succ]
          ring

  intro M
  have h_tendsto : Tendsto (fun k => $\|$w$\|$ * lam ^ k) atTop atTop := by
    apply Filter.Tendsto.const_mul_atTop hw_pos
    apply tendsto_pow_atTop_atTop_of_one_lt hlam

  rcases (Filter.tendsto_atTop_atTop.mp h_tendsto) (M + 1) with $\langle$N0, hN0$\rangle$
  use n + N0
  intro m hm
  let k := m - n
  have hnm : n $\le$ m := le_trans (Nat.le_add_right n N0) hm
  have hk : m = n + k := (Nat.add_sub_of_le hnm).symm
  rw [hk, add_comm]
  dsimp [orbit]
  rw [Function.iterate_add_apply]
  have hm' : N0 + n $\le$ m := by rwa [add_comm]
  specialize hN0 k (Nat.le_sub_of_add_le hm')
  calc $\|$orbit c w k$\|$ $\ge$ $\|$w$\|$ * lam ^ k := growth k
    _ $\ge$ M + 1 := hN0
    _ > M := lt_add_one M
\end{lstlisting}

\textbf{Explanation of \texttt{escape\_lemma}:}
\begin{itemize}
    \item \texttt{lam}: We define $\lambda = |w| - |c|/|w|$.
    \item \texttt{hlam}: We proved $\lambda > 1$ in \texttt{factor\_gt\_one}.
    \item \texttt{growth}: By induction, we show $|f^k(w)| \ge |w| \cdot \lambda^k$.
        \begin{itemize}
            \item \textbf{Base case}: For $k=0$, $|w| \ge |w| \cdot 1$.
            \item \textbf{Step}:
                \item $|f(z_k)| \ge |z_k|^2 - |c| = |z_k|(|z_k| - |c|/|z_k|)$.
                \item Since $|z_k| \ge |w|$, by monotonicity \texttt{sub\_div\_mono}, $(|z_k| - |c|/|z_k|) \ge (|w| - |c|/|w|) = \lambda$.
                \item So $|f(z_k)| \ge |z_k| \cdot \lambda$.
                \item Using IH $|z_k| \ge |w|\lambda^k$, we get $|f(z_k)| \ge |w|\lambda^k \cdot \lambda = |w|\lambda^{k+1}$.
        \end{itemize}
    \item \texttt{h\_tendsto}: Since $\lambda > 1$, the sequence $|w|\lambda^k$ tends to infinity.
    \item \textbf{Conclusion}: For any $M$, there is a time when the orbit exceeds $M$ and stays there.
\end{itemize}

\begin{proof}
Let $w=f^n(z)$ and $\lambda=|w|-|c|/|w|>1$. By induction one shows $|f^k(w)|\ge|w|\lambda^k$ for all $k$: the base case is trivial, and if $|f^k(w)|\ge|w|\lambda^k$ then
$|f^{k+1}(w)|\ge|f^k(w)|^2-|c|=|f^k(w)|(|f^k(w)|-|c|/|f^k(w)|)\ge|f^k(w)|\lambda\ge|w|\lambda^{k+1}$. Since $\lambda>1$ the right-hand side tends to infinity, so for any $M$ there exists $K$ with $|w|\lambda^K>M$, and hence for all $m\ge n+K$ we have $|f^{m}(z)|=|f^{m-n}(w)|\ge|w|\lambda^{m-n}>M$, proving escape to infinity.
\end{proof}

\subsubsection{Norm Non-Decreasing}

\begin{lstlisting}[language=Lean]
lemma norm_orbit_ge_of_norm_ge_R (c z : $\mathbb{C}$) (n : $\mathbb{N}$) (h : $\|$z$\|$ > R c) :
    $\|$orbit c z n$\|$ $\ge$ $\|$z$\|$ := by
  induction n with
  | zero => simp
  | succ n ih =>
    have h_n : $\|$orbit c z n$\|$ $\ge$ $\|$z$\|$ := ih
    have h_n_R : $\|$orbit c z n$\|$ > R c := lt_of_lt_of_le h h_n
    rw [orbit_succ]
    have : $\|$fc c (orbit c z n)$\|$ $\ge$ $\|$orbit c z n$\|^2$ - $\|$c$\|$ := norm_fc_ge_norm_sq_sub_norm_c c _
    apply le_trans h_n
    apply le_trans _ this
    have h_zn : $\|$orbit c z n$\|$ > R c := h_n_R
    have h_R : R c $\ge$ 1 + $\|$c$\|$ := R_ge_one_plus_c c
    have h_R2 : R c $\ge$ 2 := R_ge_two c
    have h_zn_gt_1 : $\|$orbit c z n$\|$ > 1 := lt_of_le_of_lt (by linarith) h_zn
    have h_zn_sub_1 : $\|$orbit c z n$\|$ - 1 > $\|$c$\|$ := by linarith
    have : $\|$orbit c z n$\|$ * ($\|$orbit c z n$\|$ - 1) > 1 * $\|$c$\|$ := by
      by_cases hc : $\|$c$\|$ = 0
      · rw [hc, mul_zero]
        apply mul_pos
        · linarith
        · linarith
      · have h_calc : 1 * $\|$c$\|$ < $\|$orbit c z n$\|$ * ($\|$orbit c z n$\|$ - 1) := calc
          1 * $\|$c$\|$ < 1 * ($\|$orbit c z n$\|$ - 1) := by
            gcongr
          _ < $\|$orbit c z n$\|$ * ($\|$orbit c z n$\|$ - 1) := by
            apply mul_lt_mul_of_pos_right
            · exact h_zn_gt_1
            · linarith
        exact h_calc
    rw [one_mul] at this
    rw [mul_sub, mul_one] at this
    linarith
\end{lstlisting}

\textbf{Explanation of \texttt{norm\_orbit\_ge\_of\_norm\_ge\_R}:}
We prove by induction that if $|z| > R(c)$, then $|f^n(z)| \ge |z|$.
\begin{itemize}
    \item \textbf{Step}: Assuming $|z_n| \ge |z|$, we need $|z_{n+1}| \ge |z_n|$ to prove monotonicity (and hence $|z_{n+1}| \ge |z|$).
    \item We know $|z_{n+1}| \ge |z_n|^2 - |c|$.
    \item We want $|z_n|^2 - |c| \ge |z_n|$, or $|z_n|^2 - |z_n| \ge |c|$, or $|z_n|(|z_n|-1) \ge |c|$.
    \item Since $|z_n| > R(c) \ge 1+|c|$, we have $|z_n|-1 > |c|$.
    \item Also $|z_n| > 2 \ge 1$.
    \item So $|z_n|(|z_n|-1) > 1 \cdot |c| = |c|$.
    \item Therefore $|z_{n+1}| > |z_n|$.
\end{itemize}

\begin{proof}
Assume $|z|>R(c)$. By induction suppose $|f^n(z)|\ge|z|$; write $z_n=f^n(z)$. Then
$|f(z_n)|\ge|z_n|^2-|c|=|z_n|(|z_n|-1)$. Since $|z_n|>R(c)\ge1+|c|$ we have $|z_n|-1>|c|$ and $|z_n|>1$, so $|z_n|(|z_n|-1)>|c|$. Hence $|f(z_n)|>|z_n|$, completing the induction and proving the sequence is non-decreasing (in fact strictly increasing).
\end{proof}
