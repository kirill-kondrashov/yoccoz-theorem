\section{The Escape Lemma}
\label{sec:Escape}

This section establishes the fundamental escape criterion for the quadratic family $f_c(z) = z^2 + c$. The result ensures that if the orbit of a point leaves a sufficiently large disk, it necessarily escapes to infinity.

\subsection{Statement}
We recall the escape radius $R(c) = \max(2, 1 + |c|)$.

The \textbf{Escape Lemma} asserts that if for some iteration $n$, the point $z_n = f_c^{\circ n}(z)$ lies outside the disk of radius $R(c)$, i.e., $|z_n| > R(c)$, then the orbit of $z$ escapes to infinity:
\[
\lim_{m \to \infty} |f_c^{\circ m}(z)| = \infty
\]

\subsection{Proof Strategy}
The proof relies on establishing a geometric growth property for points outside the escape radius.
We define a growth factor $\lambda(w)$ for any $w$ with $|w| > R(c)$:
\[
\lambda(w) = |w| - \frac{|c|}{|w|}
\]
The key steps are:
\begin{enumerate}
    \item \textbf{Growth Factor > 1}: We show that if $|w| > R(c)$, then $\lambda(w) > 1$.
    \item \textbf{Inductive Lower Bound}: We prove by induction that for all $k \ge 0$, the subsequent iterates satisfy:
    \[
    |f_c^{\circ k}(w)| \ge |w| \cdot \lambda(w)^k
    \]
    \item \textbf{Conclusion}: Since $\lambda(w) > 1$, the term $\lambda(w)^k$ tends to infinity as $k \to \infty$, forcing the orbit to escape.
\end{enumerate}

\subsection{Significance for MLC}
This lemma is crucial for the construction of the Yoccoz puzzle:
\begin{itemize}
    \item It defines the \textbf{Basin of Infinity} $A(\infty) = \{ z \mid f_c^{\circ n}(z) \to \infty \}$, which contains the complement of the closed disk $\overline{D}(0, R(c))$.
    \item It allows the definition of the \textbf{Green's Function} on $A(\infty)$, which provides the "grid" (equipotentials and external rays) for the puzzle partition.
\end{itemize}
