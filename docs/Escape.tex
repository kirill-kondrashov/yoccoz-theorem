\section{The Escape Lemma}
\label{sec:Escape}

This section establishes the fundamental escape criterion for the quadratic family $f_c(z) = z^2 + c$. The result ensures that if the orbit of a point leaves a sufficiently large disk, it necessarily escapes to infinity.

\subsection*{Escape Radius}
Recall the definition of the escape radius from the Basic properties:
\[ R(c) = \max(2, 1 + |c|). \]
This radius provides a threshold: any point outside the disk $D(0, R(c))$ will have an orbit that grows without bound.

\begin{figure}[ht]
    \centering
    \includegraphics[width=0.6\textwidth]{images/escape_radius.png}
    \caption{The Escape Radius $R(c)$. Orbits starting outside the red circle (blue trajectory) escape to infinity. Orbits inside (green) may remain bounded.}
    \label{fig:escape_radius}
\end{figure}

\subsection*{The Escape Lemma}
\textbf{Lemma} (\texttt{escape\_lemma}).
\textit{Let $c, z \in \mathbb{C}$. If there exists an $n \in \mathbb{N}$ such that $|f_c^n(z)| > R(c)$, then the orbit of $z$ escapes to infinity:
\[ \lim_{m \to \infty} |f_c^m(z)| = \infty. \]
}

\begin{proof}[Proof Idea]
(See Lean: \texttt{MLC.Quadratic.escape\_lemma})

Let $w = f_c^n(z)$. Since $|w| > R(c)$, we define a ``growth factor'' $\lambda(w)$:
\[ \lambda(w) = |w| - \frac{|c|}{|w|}. \]
First, we show that $|w| > R(c)$ implies $\lambda(w) > 1$ (Lean: \texttt{factor\_gt\_one}).
Then, using induction on $k$, we establish the inequality:
\[ |f_c^k(w)| \ge |w| \cdot \lambda(w)^k. \]
Since $\lambda(w) > 1$, the term $\lambda(w)^k$ tends to infinity as $k \to \infty$, forcing $|f_c^k(w)| \to \infty$.
\end{proof}

\subsection*{Orbit Growth}
A direct consequence of the escape mechanism is that the norm of the orbit cannot decrease once it exceeds the escape radius.

\textbf{Lemma} (\texttt{norm\_orbit\_ge\_of\_norm\_ge\_R}).
\textit{If $|z| > R(c)$, then for all $n \in \mathbb{N}$,
\[ |f_c^n(z)| \ge |z|. \]
}

\subsection*{Significance for MLC}
The escape lemma is foundational for the Yoccoz puzzle construction:
\begin{itemize}
    \item \textbf{Basin of Infinity}: It characterizes the basin of attraction of infinity, $A(\infty)$, showing it contains the exterior of the disk of radius $R(c)$.
    \item \textbf{Green's Function}: It ensures the well-definedness of the Green's function $G_c(z) = \lim_{n \to \infty} 2^{-n} \log^+ |f_c^n(z)|$ on the basin of infinity, which in turn defines the equipotentials and external rays used to cut the puzzle pieces.
\end{itemize}
